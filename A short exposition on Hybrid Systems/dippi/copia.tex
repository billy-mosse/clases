%%%%%%%%%%%%%%%%%%%%%%%%%%%%%%%%%%%%%%%%
% Beamer Presentation
% LaTeX Template
% Version 1.0 (10/11/12)
%
% This template has been downloaded from:
% http://www.LaTeXTemplates.com
%
% License:
% CC BY-NC-SA 3.0 (http://creativecommons.org/licenses/by-nc-sa/3.0/)
%
%%%%%%%%%%%%%%%%%%%%%%%%%%%%%%%%%%%%%%%%%

%----------------------------------------------------------------------------------------
%   PACKAGES AND THEMES
%----------------------------------------------------------------------------------------

\documentclass{beamer}

\mode<presentation> {

% The Beamer class comes with a number of default slide themes
% which change the colors and layouts of slides. Below this is a list
% of all the themes, uncomment each in turn to see what they look like.

%\usetheme{default}
%\usetheme{AnnArbor}
%\usetheme{Antibes}
%\usetheme{Bergen}
%\usetheme{Berkeley}
%\usetheme{Berlin}
%\usetheme{Boadilla}
%\usetheme{CambridgeUS}
%\usetheme{Copenhagen}
%\usetheme{Darmstadt}
%\usetheme{Dresden}
%\usetheme{Frankfurt}
%\usetheme{Goettingen}
%\usetheme{Hannover}
%\usetheme{Ilmenau}
%\usetheme{JuanLesPins}
%\usetheme{Luebeck}
\usetheme{Madrid}
%\usetheme{Malmoe}
%\usetheme{Marburg}
%\usetheme{Montpellier}
%\usetheme{PaloAlto}
%\usetheme{Pittsburgh}
%\usetheme{Rochester}
%\usetheme{Singapore}
%\usetheme{Szeged}
%\usetheme{Warsaw}


% As well as themes, the Beamer class has a number of color themes
% for any slide theme. Uncomment each of these in turn to see how it
% changes the colors of your current slide theme.

%\usecolortheme{albatross}
%\usecolortheme{beaver}
%\usecolortheme{beetle}
%\usecolortheme{crane}
%\usecolortheme{dolphin}
%\usecolortheme{dove}
%\usecolortheme{fly}
%\usecolortheme{lily}
%\usecolortheme{orchid}
%\usecolortheme{rose}
%\usecolortheme{seagull}
%\usecolortheme{seahorse}
%\usecolortheme{whale}
%\usecolortheme{wolverine}

%\setbeamertemplate{footline} % To remove the footer line in all slides uncomment this line
%\setbeamertemplate{footline}[page number] % To replace the footer line in all slides with a simple slide count uncomment this line

%\setbeamertemplate{navigation symbols}{} % To remove the navigation symbols from the bottom of all slides uncomment this line
}
\usepackage[utf8]{inputenc}
\usepackage{amsmath}
\usepackage{graphicx}
\graphicspath{ {images/} }
\usepackage[all,cmtip]{xy}
\usepackage{hyperref}
\usepackage{graphicx} % Allows including images
\usepackage{booktabs} % Allows the use of \toprule, \midrule and \bottomrule in tables

%----------------------------------------------------------------------------------------
%   TITLE PAGE
%----------------------------------------------------------------------------------------

\title{Hybrid systems: an introduction} % The short title appears at the bottom of every slide, the full title is only on the title page

\author{Billy} % Your name
\institute[UBA] % Your institution as it will appear on the bottom of every slide, may be shorthand to save space
{
Universidad de Buenos Aires\\ % Your institution for the title page
\medskip
\textit{billy.mosse@gmail.com} \\ % Your email address


\tiny{Robé el template de internet} \\

}
\date{\today} % Date, can be changed to a custom date

\begin{document}

\begin{frame}
\titlepage % Print the title page as the first slide
\end{frame}

\begin{frame}
\frametitle{Overview} % Table of contents slide, comment this block out to remove it
\tableofcontents % Throughout your presentation, if you choose to use \section{} and \subsection{} commands, these will automatically be printed on this slide as an overview of your presentation
\end{frame}

%----------------------------------------------------------------------------------------
%   PRESENTATION SLIDES
%----------------------------------------------------------------------------------------

%------------------------------------------------
\section{Introduction} % Sections can be created in order to organize your presentation into discrete blocks, all sections and subsections are automatically printed in the table of contents as an overview of the talk
%------------------------------------------------

%\subsection{Subsection Example} % A subsection can be created just before a set of slides with a common theme to further break down your presentation into chunks

%------------------------------------------------

\subsection{A little background}
\begin{frame}

\frametitle{A little background}

Before:
\begin{itemize}
	\item Rapid progress in hardware and software $\Rightarrow$ 
ambitious projects $\Rightarrow$ costly ad-hoc integration and validation of systems. (Bottleneck)
	\item Centralized control still worked over a distributed system: each new increment was slowly made, after extensive (but not exhaustive) tests
	\item Ran with considerable "slack" (and yet still failed)
\end{itemize}

This centralized control, ad-hoc, slack-y approach can't meet today's standards.

We need a change of paradigm for distributed control that
\begin{itemize}
	\item avoids the high communication and computation costs of a central control
	\item but permits central authority
\end{itemize}

So...a hierarchical structure, may be? \newline


\tiny{"Studies [citation needed] indicate that, if there is no change to the structure of Air Ttraffic Control, then by the year 2015 there could be a major accident every 7 to 10 days."}
\end{frame}
%------------------------------------------------


\begin{frame}
	\frametitle{Hierarchical structure}

	\centerline{Discrete controls of the system (high level) }
	\centerline{COMMANDS $\Bigg\downarrow\Bigg\uparrow$ INFORMATION}
	
	\centerline{Continuous control laws of the world (low level)}
\end{frame}


%\begin{frame}
%	\frametitle{Problems we could resolve}
%\begin{itemize}
%	\item Automated highway systems
%	\item Air traffic control
%	\item Groups of anmanned aerial vehicles
%	\item Underwater autonomous vehicles
%	\item Mobile offshore platforms
%
%\end{itemize}
%\end{frame}


\begin{frame}
	\frametitle{Literature on hybrid systems}
	 One idea: extend techniques for finite state automata to include systems with simple continuous dynamics. \newline
	 
	 How? Model checking and/or deductive theorem proving. \\
	 Emphasis: computability, decidability. Is "Does the problem satisfy the specification" decidable? \newline
	 
	Models and decidability
results have been obtained for timed automata [1], linear hybrid automata [2], and hybrid
input/output automata [3]. \newline

Decidability results for linear hybrid automata are
fairly narrow.  For all but the simplest continuous linear dynamics (two-dimensional rect-
angular differential inclusions), reachability properties are semi-decidable at best, and in
most cases undecidable. See \href{<https://www.irif.fr/~asarin/papers/incl.pdf>}{[4]}, Section 2, for more details.
\end{frame}


\begin{frame}
\frametitle{Literature on hybrid systems}

Un/decidability sometimes may be proven by (bi)simulation

\begin{itemize}
	\item In dynamical systems: with topological equivalence and homomorphism
	\item In hybrid systems:
		\begin{itemize}
			\item Not easy
			\item Simulation is an ad hoc definition that must include reachability: if system A simulates system B then it provides a reduction from reachability problem for A to the one for B
		\end{itemize}
\end{itemize}

In the end, it's another reduction to Halt. 
(See \href{<http://www.sciencedirect.com/science/article/pii/030439759400228B>}{[5]}
. Also, \href{http://www.sciencedirect.com/science/article/pii/S0890540112000028}{[6]}.
Interestingly, they use some tools from Topology and Geometry.)\end{frame}


\begin{frame}
	\frametitle{Literature on hybrid systems}
	 Another idea:  extend the standard modeling, reachability and stability
analyses, and controller design techniques in continuous state space and continuous time dynamical systems and control to capture the interaction between the continuous
and discrete dynamics. \newline

Goal: extend standard modeling, reachability* and stability analyses, and controller design techniques. \newline

* Really hard for systems whose dynamics are nonlinear or are of order greater than one.  Only recently [$<$2007], some attempts to directly
approach this problem have been reported in the literature. TODO: check if it is still hard \newline

%TODO remove/add references from pdf here

Tools extended:  stability theory [17], optimal control [17, 53, 67], and control of discrete event systems [49, 38] \newline

% Sampling of continuous control to convert it to discrete control

\end{frame}

\begin{frame}
	Stability theory addresses the stability of solutions of differential equations and of trajectories of dynamical systems under small perturbations of initial conditions. \newline
	
	
	See \href{<http://www.mit.edu/people/mitter/SKM_theses/95_6_Branicky_DSc.pdf>}{[1]} or maybe \href{<http://ieeexplore.ieee.org/stamp/stamp.jsp?arnumber=664149>}{[2]} for the extension to hybrid systems.
\end{frame}

\begin{frame}
	\frametitle{Optimal control}
	x $\leftarrow$ state \newline
	u $\leftarrow$ controllable parameters \newline
	minimize 	$J = \psi[x(T)] + \int_{0}^{T} \ell(u,x(t)) dt$
	
\end{frame}

\begin{frame}
	One solution (for the continuous problem):
	1) discretization of the problem
	2) Principle of Optimality: An optimal policy has the property that whatever the initial state and initial decision are, the remaining decisions must constitute an optimal policy with regard to the state resulting from the first decision. (See Bellman, 1957, Chap. III.3.)
	3) Go backwards.
	Problems: costly; "curse of dimensionality"
	4) http://www.mpi-magdeburg.mpg.de/94992/Optimal-Control-of-Hybrid-Systems
\end{frame}

\begin{frame}
	Another solution: gradient descent?
\end{frame}

\begin{frame}
	Another solution: perturbate ($u$) + differentiate, and get necessary conditions (stated in
%TODO add link
 %https://en.wikipedia.org/wiki/Pontryagin%27s_maximum_principle
	
%TODO add link
	See %http://www.bauer.uh.edu/rsusmel/phd/MR-15.pdf for some nice examples.
	(Or let's do together the "a minimizing curve in the plane is a straight line) % - 1 + f'^2
	% Remember the fundamental lemma of calculus of variations
\end{frame}

%TODO algo
\begin{frame}
 How do we solve this in the hybrid case?
\begin{itemize}
	\item Get a lower bound for the minimum. See \href{<http://citeseerx.ist.psu.edu/viewdoc/download?doi=10.1.1.66.8562&rep=rep1&type=pdf>}{[8]} (uses a discretization of Bellman's inequality to be able to do linear programming)
	\item Pressing a discrete switch every time we'd get an at-the-moment improvement gives us an upper bound for the minimum.
\end{itemize} 
 
\end{frame}




\begin{frame}
\frametitle{What's interesting}
\begin{itemize}
	\item Continuity with respect to initial conditions (for simulations)
	\begin{itemize}
		\item Using topology (homotopy theory) tools (Sokorod topology for paths) - not practical 
		(See \href{https://www.researchgate.net/publication/3792882_Regularity_of_solutions_and_homotopic_equivalence_for_hybrid_systems}{[9]})
		\item \href{https://www.researchgate.net/profile/Karl_Johansson2/publication/2920336_Dynamical_Properties_of_Hybrid_Automata/links/02e7e51aa0db91ccdc000000.pdf}{[10]}
		is a more practical (and concrete?) but still limited method	
	\end{itemize}
	\item Continuity with respect to initial condition \textbf{parameters}
	\item Well-posedness (existence and uniqueness of solutions) %TODO falta la bibliografia
	\item Zeno executions
\end{itemize}


\end{frame}

%------------------------------------------------


%------------------------------------------------
\subsection{Problems}

\begin{frame}
\frametitle{Zeno executions}

An execution is called Zeno, if it contains an infinite number of transitions in a finite
amount of time (Achilles vs Tortoise \href{https://en.wikipedia.org/wiki/Zeno's_paradoxes}[11]). An automaton is called Zeno if it accepts a Zeno execution. \newline

Other examples: bouncing ball; 2 water tanks w/alternating faucet.

\includegraphics[scale=0.3]{bouncing_ball.png}


Problems: 
\begin{itemize}
	\item Semantical: how is it defined beyond the Zeno time?
	\item Analysis: induction and reachability proofs become suspect
	\item Controller synthesis: it can cheat by forcing time to converge
	\item Simulation: it stalls at Zeno time
\end{itemize}

\end{frame}

%------------------------------------------------



%------------------------------------------------


\begin{frame}
	\frametitle{Resolving the Zeno phenomenon}
	The only known conditions to characterize the Zeno phenomenon are fairly trivial. \newline
	
	Possible solution: a regularization approach (of the automata), inspired by the method used in differential equations. (\href{http://cms.dm.uba.ar/depto/public/grado/fascgrado7.pdf}{[12]}, 7.4.3) \newline
	
	E.g. 
	\begin{itemize}
		\item Water tank automaton: the switch takes $\epsilon$ to activate.
		\item Bouncing ball: each bounce takes $\epsilon$ (meanwhile, gravity still applies) 		
	\end{itemize}
	
	As $\varepsilon \rightarrow 0$, then the non-Zeno automata $H_\varepsilon $ converges to the original Zeno automata (the states converge, in the Skorohod metric, i.e., wiggling space and time a bit) \newline
	
	%($\phi_\varepsilon : Q_\varepsilon \times X_\varepsilon \rightarrow Q \times X$ converges in the Skorohod metric)
		
	 See \href{http://users.ece.gatech.edu/magnus/Papers/ZenoAutomata.pdf}{[13]} for more.
			
\end{frame}

\subsection{Some definitions}

\begin{frame}
	\frametitle{Some basic descriptions: Timed automata}
	Timed automata:
	\begin{itemize}
		\item Nice augmentation of $\omega$-automata (Seen in class)
		\item Words are timed (each letter is presented at a certain time)
		\item Time satisfies monotonicity and progress
		\item Clocks that can be reseted (though not in the language)
		\item You can check time
		\item Reachability and eventuality properties are decidable
	\end{itemize}
	
	\centerline{\includegraphics[scale=0.9]{timed_automata.png}}
	
	See \href{www.cis.upenn.edu/~alur/TCS94.pdf}{[13]} and \href{https://en.wikipedia.org/wiki/Timed_automaton}[14] for more fun on timed automata.
\end{frame}

\begin{frame}
	\frametitle{Some basic descriptions: other automatas}
	
	\begin{itemize}
		\item Linear hybrid automata: models $A\dot{x}\leq b$. Decidability results are fairly narrow. TODO: research how do they model discrete actions.
		\item Hybrid input/output automata. See
		%TODO %http://citeseerx.ist.psu.edu/viewdoc/download?doi=10.1.1.13.1457&rep=rep1&type=pdf and perhaps http://www.ita.cs.ru.nl/publications/papers/fvaan/hioaslides.pdf
	Permits:
	\begin{itemize}
		\item Easy decomposition of description and analysis
		\item Showing that one automata implements another one
		\item Showing that an automata doesn't contribute to produce Zeno behaviour. This property, under some compatibility conditions, is preserved by composition.
	\end{itemize}
		
	\end{itemize}
\end{frame}



\begin{frame}
\frametitle{The language}
	The modelling language must be
\begin{itemize}
	\item descriptive: to model how discrete evolution affects and is affected by continuous evolution, to allow non-deterministic models to capture uncertainty
	\item composable
	\item abstractable, to refine problems down and compose results up
\end{itemize}

\end{frame}

\begin{frame}
	\frametitle{One of the languages: Hybrid Automata}
	$H = (Q, X, Init, f, Inv/Dom, E, G, R)$ %Inv y Dom son intercambiables en el libro
	
	\begin{itemize}
		\item $Q$ (countable) is a set of discrete variables
		\item $X$ is a set of continuous variables
		\item $Init \subset Q \times X$ is a set of initial states
		\item $f: Q \times X \rightarrow TX$ is a vector field: it usually describes the derivative of the continuous variable
		\item $Inv/Dom: Q \rightarrow P(X)$ assigns to each $q \in Q$ an invariant set
		\item $E \subset Q \times Q$ is a collection of discrete transitions
		\item $G : E \rightarrow P(X)$ assigns to each $e=(q,q') \in E$ a guard.
		\item $R : E \times X \rightarrow P(X)$ assigns to each $e = (q,q') \in E$ and $x \in X$ a reset relation
	\end{itemize}
	
Example: Water Tank System. Two tanks are leaking are at constant rates ($v_1,v_2$) respectively. Water is added constantly though a hose controlled by an instantaneous switch.

¿Can the water of both tanks ($x_1,x_2$) be kept above ($r_1,r_2$)?
\end{frame}

\begin{frame}
\includegraphics[scale=0.8]{tank_system.png}
\end{frame}

\begin{frame}
	\frametitle{Definition: Hybrid time set.}
	
	A hybrid time set is a set $\tau = \{I_0,\cdots,I_N\}$ (finite or infinite) of almost-disjoint ordered intervals.
	\begin{itemize}
		\item $I_i = [\tau_i, \tau'_i]$ with $\tau'_i = \tau_{i+1}\ \forall\ i < N$
		\item If $N < \infty$ then $I_N$ might be $[\tau_N, \tau'_N)$
	\end{itemize}	 
	\includegraphics[scale=0.7]{partition.png}
\end{frame}

\begin{frame}
	\frametitle{Definition: execution}
	
	Definition: A hybrid trajectory is a triple ($\tau,q,x$), with the hybrid time set $\tau = \{I_i\}$, and two sequences of functions $q = \{q_i\}, x = \{x_i\}$, with $q_i(.) : I_i \rightarrow Q,x_i(.) : I_i \rightarrow \mathbb{R}^n$
	
	Definition: an execution $\mathbb{X}$ of a hybrid automaton $H$ is a hybrid trajectory ($\tau,q,x$) satisfying:
	\begin{itemize}
		\item Initial condition: $q(0),x(0) \in Init$
		\item Discrete evolution: the discrete transitions must be valid and the guards and reset functions must be satisfied
		\item Continuous evolution: $\frac{dx_i}{dt} = f(q_i(t),x_i(t))$, and between them they must belong to the correct domain. $q_i$ must be constant over $I_i$
	\end{itemize}
\end{frame}

\begin{frame}
	\frametitle{Controllers: a really brief introduction}
	
	
	%TODO AUTOMATA ES PLURAL, AUTOMATON ES SINGULAR. ARREGLAR
	
	%TODO  arreglar detalles de esto
	\begin{itemize}
		\item We add input variables $v \in V$ (continuous/discrete and controls(U)/disturbances (D)). Controls are used to guide the continuous evolution through $f$, enable internal transitions through $G$, force internal transitions through $Dom$, and determine the state after a transition through $R$.
		\item Same with output variables (for each state of the automaton)
		%(T \times Q \times X)^*
		\item $C : \mathbb{X}^* \rightarrow 2^U$ a controller that restricts the control input variables allowed at the final state
		\item $\mathbb{H}_C$ the set of "closed loop causal executions" (executions with inputs always approved by $C$)
		\item $C$ satisfies ($Q \cap X, \square F$) if $\square F(\chi) = True\ \forall\ \chi \in \mathbb{X}$. Problems: Zeno, blocked executions		
		\item We say a controller is memoryless if every time two executions $\chi_1,\chi_2$ have the same endpoint, $C(\chi_1) = C(\chi_2)$
	\end{itemize}
\end{frame}

\begin{frame}
(\href{https://pdfs.semanticscholar.org/6c3c/c35e9e723a586390244565cfa0081cc86f7e.pdf}{[0]}, Proposition 6.2) \newline
	
	
	A controller satisfying	 ($Q \cap X, \square F$) exists if and only if a memorless controller satisfying ($Q \cap X, \square F$) exists. \newline

Idea of the proof: a drawing. Also, we must strongly use that valid executions and properties are "locally memoryless"

(By contradiction)

1. Assume $\chi_1,\chi_2$, reaching $(x,q)$ with $C(\chi_1) \neq C(\chi_2)$.\\
2. Then, append $\chi'$ to $\chi_2$, but with a control rule applied to $\chi_1 \chi'$ so that $F$ doesn't hold somewhere. (We can assume we can do it for some pair $\chi_1,\chi_2$, because we are supposing a memoryless strategy doesn't exist)\\
3. $F$ breaks down for $\chi_2 \chi'$ at some time $t$. $\chi_1 \chi'$ is a valid execution as executions are memoryless, so the same happens at time $t$. Absurd! Because $\chi_1 \chi'$ was also validated by the controller, by construction, and the controller satisfied ($Q \cap X, \square F$)

\end{frame}




\section{Bibliografía}

\begin{frame}
\frametitle{Bibliografía}

\tiny{
1. R. Alur and D. Dill.  A theory of timed automata. Theoretical  Computer Science,
126:183–235, 1994. \newline

2. R. Alur, C. Courcoubetis, T. A. Henzinger, and P. H. Ho. Hybrid automaton: An algorithmic approach to the specification and verification of hybrid systems. In Robert L. Grossman, Anil Nerode, Anders P. Ravn, and Hans Rischel, editors, Hybrid Systems, number 736 in LNCS, pages 209–229. Springer-Verlag, Berlin, 1993. \newline

3.  N. Lynch, R. Segala, F. Vaandrager, and H.B. Weinberg.  Hybrid I/O automata.  In
Hybrid Systems III, number 1066 in LNCS, pages 496–510. Springer-Verlag, Berlin, 1996. \newline}

\end{frame}

%
%\begin{frame}
%\frametitle{Multiple Columns}
%\begin{columns}[c] % The "c" option specifies centered vertical alignment while the "t" option is used for top vertical alignment
%
%\column{.45\textwidth} % Left column and width
%\textbf{Heading}
%\begin{enumerate}
%\item Statement
%\item Explanation
%\item Example
%\end{enumerate}
%
%\column{.5\textwidth} % Right column and width
%Lorem ipsum dolor sit amet, consectetur adipiscing elit. Integer lectus nisl, ultricies in feugiat rutrum, porttitor sit amet augue. Aliquam ut tortor mauris. Sed volutpat ante purus, quis accumsan dolor.
%
%\end{columns}
%\end{frame}
%
%
%
%%------------------------------------------------
%\section{Second Section}
%%------------------------------------------------
%
%\begin{frame}
%\frametitle{Table}
%\begin{table}
%\begin{tabular}{l l l}
%\toprule
%\textbf{Treatments} & \textbf{Response 1} & \textbf{Response 2}\\
%\midrule
%Treatment 1 & 0.0003262 & 0.562 \\
%Treatment 2 & 0.0015681 & 0.910 \\
%Treatment 3 & 0.0009271 & 0.296 \\
%\bottomrule
%\end{tabular}
%\caption{Table caption}
%\end{table}
%\end{frame}
%
%%------------------------------------------------
%
%\begin{frame}
%\frametitle{Theorem}
%\begin{theorem}[Mass--energy equivalence]
%$E = mc^2$
%\end{theorem}
%\end{frame}
%
%%------------------------------------------------
%
%\begin{frame}[fragile] % Need to use the fragile option when verbatim is used in the slide
%\frametitle{Verbatim}
%\begin{example}[Theorem Slide Code]
%\begin{verbatim}
%\begin{frame}
%\frametitle{Theorem}
%\begin{theorem}[Mass--energy equivalence]
%$E = mc^2$
%\end{theorem}
%\end{frame}\end{verbatim}
%\end{example}
%\end{frame}
%
%%------------------------------------------------
%
%\begin{frame}
%\frametitle{Figure}
%Uncomment the code on this slide to include your own image from the same directory as the template .TeX file.
%%\begin{figure}
%%\includegraphics[width=0.8\linewidth]{test}
%%\end{figure}
%\end{frame}
%
%%------------------------------------------------
%
%\begin{frame}[fragile] % Need to use the fragile option when verbatim is used in the slide
%\frametitle{Citation}
%An example of the \verb|\cite| command to cite within the presentation:\\~
%
%This statement requires citation \cite{p1}.
%\end{frame}
%
%%------------------------------------------------
%
%\begin{frame}
%\frametitle{References}
%\footnotesize{
%\begin{thebibliography}{99} % Beamer does not support BibTeX so references must be inserted manually as below
%\bibitem[Smith, 2012]{p1} John Smith (2012)
%\newblock Title of the publication
%\newblock \emph{Journal Name} 12(3), 45 -- 678.
%\end{thebibliography}
%}
%\end{frame}
%
%%------------------------------------------------
%
%\begin{frame}
%\Huge{\centerline{The End}}
%\end{frame}
%
%%----------------------------------------------------------------------------------------

\end{document}	