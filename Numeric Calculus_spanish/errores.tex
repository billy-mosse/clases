\documentclass[10pt,a4paper,final]{report}
\usepackage{amsmath}
\usepackage{amsfonts}
\usepackage{listings}
\usepackage{graphicx} % Required for including images
\usepackage[font=small,labelfont=bf]{caption} % Required for specifying captions to tables and figures
\usepackage{hyperref}
\usepackage[utf8]{inputenc}
\usepackage[spanish]{babel}
\usepackage{amsthm}
\usepackage{graphicx}
\newtheorem{theorem}{Teorema}
\newtheorem{lemma}{Lema}
\newtheorem{definition}{Definición}
\newtheorem{proposition}{Proposición}
\newtheorem{observation}{Observación}
\newtheorem{corollary}{Corolario}
\newtheorem{example}{Ejemplo}

\newtheorem{caution}{¡Cuidado!}


\def\Q{\mathbb{Q}}
\def\R{\mathbb{R}}
\def\C{\mathbb{C}}
\def\I{[a,b]} 


\begin{document}

\begin{itemize}
	\item En la expresión (4.7) de la página 81 (método de la secante), donde se define $g(x)$, falta un paréntesis. Debería ser:
$g(x) = f(x) - \Bigg(f(a) + f[a,b] (x-a) + f[a,b,c] (x-a) (x-b)\Bigg)$ ya que definimos a la función para que en $a,b,$ y $c$ de $0$.
	\item La figura 4.3 de la página 70 (método de Newton-Raphson) está mal. Observar que desde $x_0$ se está yendo por la tangente hasta intersecar el gráfico de la función y no el eje $x$.
	%\item En la demostración del teorema 4.11 de la página 76 (método de Newton-Raphson), se asume que $f$ es monótona porque de todas maneras la iteración de Newton nunca irá a la izquierda. Esto no lo señalo como error, pero no estoy de acuerdo con la asumpción. En primer lugar, la función $f(x) = x^2$ es convexa pero no monótona. El algoritmo converge pero $f'(r) = f'(0) = 0$. Me pregunté si lo que se insinuaba era que podían cambiar la función $f$ por otra que sea monótona que la interpole en los puntos $x_i$ pero sabemos interpolar finitos puntos, no infinitos. Update: revisé la cursada de este cuatrimestre y mencionan que $f(x) = x^2-4$ cumple que aunque la función es convexa, si empezás en $x_0=0$, como $f'(0)=0$, el algoritmo no anda.
	 \item En la página 83, al final de la demostración del orden de convergencia del método de la secante se afirma que el punto fijo es $\bar{x} = c_\infty^\frac{1}{p}$. Pero al reemplazar en la ecuación no da. Yo al resolver la ecuación obtuve $\bar{x}= c_\infty^{\frac{p}{p+1}}$
	\item En la página 71 el error de Newton Raphson se define como $e_{n+1} = x_{n+1} -r$, pero para el método de la secante se define al revés, $e_{n+1} = r - x_{n+1}$. No es realmente un error, uno lo define como quiere, pero puede confundir a la hora de hacer cuentas (¿si uno prueba que el error es positivo lo intuitivo no sería que la sucesión esté a la derecha de la raíz?)
	\item En la página 174, en el ejemplo 8.5, hay un error de tipeo. Donde dice ``para aproximar $\sqrt{2}$"\ debería decir ``para aproximar $\sqrt{t}$"\ (la solución de la ecuación diferencial).
\end{itemize}

\end{document}
