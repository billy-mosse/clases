\documentclass[10pt,a4paper]{article}
\usepackage[utf8]{inputenc}
\usepackage{amsmath}
\usepackage{amsfonts}
\usepackage{amssymb}

\usepackage{amsthm}


\usepackage[utf8]{inputenc}
\usepackage[spanish]{babel}
\usepackage{amsmath}
\usepackage{amsfonts}
\usepackage{amssymb}
\usepackage{dirtytalk}
 \usepackage{relsize}
\usepackage{mathtools}
\author{Guillermo (Billy) Mosse}
\title{(No) expresabilidad en primer orden}

\usepackage{thmtools}
\declaretheorem[name=Teorema]{theorem}
\declaretheorem[name=Lema, sibling=theorem]{lemma}
\declaretheorem[name=Definici\'on, sibling=theorem]{definition}
\declaretheorem[name=Chiste, sibling=theorem]{joke}
\declaretheorem[name=Intento de definici\'on, sibling=theorem]{trydefinition}
\declaretheorem[name=Proposici\'on, sibling=theorem]{proposition}
\declaretheorem[name=Observaci\'on, sibling=theorem]{observation}
\declaretheorem[name=An\'ecdota, sibling=theorem]{anecdote}
\declaretheorem[name=Corolario, sibling=theorem]{corollary}
\declaretheorem[name=Ejemplo, style=remark, sibling=theorem]{example}
\declaretheorem[name=Ejemplos, style=remark, sibling=theorem]{examples}
\declaretheorem[name=Advertencia, sibling=theorem]{caution}
\declaretheorem[name=Nota de color, style=remark, sibling=theorem]{remark}
\def\key#1{\{#1\}}
\def\N{\mathbb{N}}
\def\bs{\mathlarger{\mathlarger{\square}}}

\usepackage{enumitem}

\usepackage[pdftex,
            pdfauthor={Guillermo Mosse},
            pdftitle={(No) expresabilidad en primer orden},
            pdfsubject={No expresabilidad, primer orden, Compacidad},
            pdfkeywords={compacidad, expresabilidad, primer orden, lógica},
            pdfproducer={Latex with hyperref},
            pdfcreator={pdflatex}]{hyperref}

\def\l{\mathcal{L}}
\def\m{\mathcal{M}}
\def\p{\psi}
\def\vdd{\vDash}
\def\vd{\vdash}
\def\k{\mathcal{K}}


\newenvironment{solution}
  {\renewcommand\qedsymbol{$\blacksquare$}\begin{proof}[Solución]}
  {\end{proof}}
  
\newtheorem{xca}{Ejercicio}
\addtocounter{xca}{-1}

\begin{document}


\title{(No) expresabilidad en primer orden}
\author{Guillermo Mosse}

%TODO agregar la carátula y hacer que empiece en cero.
\maketitle
%\setcounter{page}{1}

En la práctica 6 había varios ejercicios en los que tenían que probar que los elementos de algún modelo eran \emph{distinguibles}, o que una relación era \emph{expresable}, o que una clase de modelos era \emph{definible}. Son tres palabras distintas pero son cosas que se entienden en contexto, y generalmente los ejercicios salían definiendo alguna fórmula $\varphi$ que cumpliera algunas cositas. (Distinguir el elemento, una relación,etc).

La clase de hoy se llama \say{No expresabilidad en primer orden}. Vamos a ver cómo probar que una propiedad \textbf{no} es expresable.

\begin{definition}
Dado $\l$ un lenguaje de primer orden, decimos que una propiedad es \emph{expresable} si existe una fórmula $\varphi$ tal que:

\begin{equation*}
\m \vDash \varphi \Leftrightarrow \m \in \k
\end{equation*}
\end{definition}

(o sea, $\m$ hace verdadera a $\varphi$ si y solo si $\m$ está en $\k$).

donde $\k$ es la clase de modelos que cumplen dicha propiedad. \bigskip

Un ejemplo de juguete donde la respuesta es afirmativa:

\begin{xca}
Sea $\l$ un lenguaje de primer orden con igualdad y un símbolo de predicado unario $Q$. Decidir si es expresable la propiedad \say{El predicado unario $Q$ vale para todo elemento del modelo.}
\end{xca}

\begin{solution}
Respuesta: sí.

$\varphi = (\forall x) P(x)$.
\end{solution}

Probar que algo \textbf{no} se puede hacer es bastante distinto a probar que algo se puede hacer. Fíjense que para ver que una propiedad es expresable, hay que encontrar una $\varphi$ que cumpla lo querido. Esto puede ser desafiante y requerir creatividad, claro está. Pero, ¿cómo hace uno para probar que algo no es expresable?

\begin{anecdote}
Una vez estaba en la Usina del Arte, en una actividad con pibes de primaria, y había un rompecabezas que era imposible de armar. Y yo les preguntaba a los chicxs por qué no les salía. En algún momento les decía que efectivamente era imposible armar el rompecabezas - después de hacerlos sufrir un rato - y les preguntaba por qué creían que era imposible. Los pibes me decían \say{porque intenté muchas veces}.

No sean como esos pibes. A veces nosotros querremos \emph{demostrar } que no existe tal $\varphi$. Para eso, vamos a suponer que existe y llegar a un absurdo.
\end{anecdote} \bigskip

\begin{xca}Sea $\l$ un lenguaje de primer orden con igualdad y un símbolo de predicado unario $Q$. Decidir si es expresable la propiedad \say{El predicado unario $Q$ solo vale para \emph{finitos} elementos del modelo.}
\end{xca}

\begin{solution}
A ver, a ver...supongamos que \emph{es} expresable. ¿Qué pinta tendrá la fórmula $\varphi$? Podemos crearnos las siguientes fórmulas:

\begin{align*}
\varphi_0 &: \neg (\exists x) Q(x)\\
\varphi_1 &: \neg \big{(}(\exists x, y) x\neq y \land Q(x) \land Q(y) \big{)}\\
\varphi_2 &: \neg \big{(}(\exists x,y,z) x\neq y \land x \neq z \land y\neq z \land Q(x) \land Q(y) \land Q(z) \big{)}\\
&\dots
\end{align*}

Si logramos crear fórmulas $\varphi_k$ que expresen \say{La propiedad $Q$ solo vale para $k$ elementos distintos}, entonces $\varphi$ tal vez podría ser:
 $$\varphi_1 \lor \varphi_2 \lor \dots$$
  o tal vez:
  $$\underbrace{\varphi_1 \lor \varphi_2 \lor \dots \lor \varphi_k}_{{\exists k}}$$
  
  Pero esto es cualquier barrabasada; ninguna de las fórmulas está bien fundada. No vale poner puntos suspensivos (las fórmulas deben ser finitas) ni llavecitas debajo (porque es cualquiera).
  
  Tal vez esto nos debería hacer sospechar que la propiedad NO va a ser expresada en primer orden.
  
  Ahora bien, de ninguna manera el ejercicio termina acá. Notemos que NO hemos probado que la fórmula $\varphi$ \textbf{debe} ser de la pinta $\varphi_1 \lor \varphi_2 \lor \dots$
  
  Tampoco podemos decir \say{intentamos expresarlo y no nos salió, así que no es expresable}. Necesitamos demostrar formalmente que la propiedad no es expresable.
  
  Probar que una propiedad sí es expresable requiere solamente dar la fórmula (y probar que sirve, claro está). Probar que no es expresable requiere algo más de trabajo. Hoy vamos a aprender una técnica de demostración que nos va a servir exactamente para esto. Son 4 pasos bien cortitos\footnote{Bueno, no tan cortitos.}:
  
  \begin{enumerate}
  	\item \textbf{Suponemos que la propiedad es expresable.} Sea $\varphi$ la fórmula de primer orden que describe la propiedad.
  	\item \textbf{Negamos incrementalmente la fórmula $\varphi$}. Creamos fórmulas $\p_1, \p_2, \dots$ que expresen lo siguiente:
  	\begin{itemize}
  		\item $\p_1$: \say{Hay un elemento que cumple $Q$.}
  		\item $\p_2$: \say{Hay dos elementos que cumple $Q$.}
  		\item $\p_2$: \say{Hay tres elementos que cumple $Q$.}
  		
  	\end{itemize}
  
  \item Definimos el conjunto $\Gamma = \key{\p_i : i \in \N} \cup \key{\varphi}$. Este conjunto \say{pide} que para cada $i \in \N$, haya $i$ elementos que cumplan $Q$, ¡pero además, como $\varphi$ pertenece al conjunto, pide que sólo finitos cumplan la propiedad!
  
  	Notemos que cada $\p_i$ está negando a $\varphi_i$ (del comienzo de la resolución). Si a $\varphi$ la pensamos como $\varphi_1 \lor \varphi_2 \lor \dots$, la negación incremental consiste en meter en $\Gamma$ a las fórmulas $\neg \varphi_1, \neg \varphi_2, \dots$
  
  \item 
	\begin{enumerate}[label=\alph*)]
		\item Probamos que $\Gamma$ es insatisfacible (eso sale solo, por la definición del conjunto).
		\item Probamos, usando Compacidad, que $\Gamma$ es satisfacible, para llegar a un absurdo. El absurdo provino de asumir que la propiedad era expresable.
	\end{enumerate}  
\end{enumerate}

Entonces, ahora en detalle:

\begin{enumerate}
	\item \textbf{Suponemos que la propiedad es expresable.} Sea $\varphi$ la fórmula de primer orden que describe la propiedad.
  	\item \textbf{Negamos incrementalmente la fórmula $\varphi$}. Creamos fórmulas $\p_1, \p_2, \dots$ que expresen lo siguiente:
  	
  Hay que definir las $\p_i$, y para eso podemos usar una función todosDistintos$_k(x_1,\dots,x_k)$ por cada $k\in \N$:

\begin{align*}
todosDistintos_k(x_1,\dots,x_k) := &x_1 \neq  x_2 \land \dots \land x_1 \neq x_k \land\\
&x_2 \neq x_3 \land \dots x_2 \neq x_k \land\\
&\land \dots \land\\
&x_{k-1} \neq x_k
\end{align*}

(Para pensar: ¿por qué está \textbf{bien} poner puntos suspensivos en la definición de $todosDistintos_k$?)

%\say{¿Alguien me puede decir por qué está bien poner puntos suspensivos en la definición de todosDistintos?} \bigskip

\begin{itemize}
	\item $\p_1: (\exists x_1) Q(x)$
	\item $\p_2: (\exists x_1, x_2)\ todosDistintos_2(x_1,x_2) Q(x_1) \land Q(x_2)$
	\item \dots
	\item $\p_n : (\exists x_1,\dots,x_n)\ todosDistintos_n(x_1,\dots,x_n) \land P(x_1) \land \dots \land P(x_n)$
	\item $\dots$
\end{itemize}


  \item 
  \begin{enumerate}[label=\alph*)]
  	\item Veamos que $\Gamma$ es insatisfacible.\footnote{Esto debería ser fácil, pues $\Gamma$ \say{pide} que solo \emph{finitos} elementos cumplan $Q$, pero también \emph{infinitos}. Y esto es ridículo, casi diríamos que...\emph{absurdo}.} Probemos esto por el absurdo.
  	 \textbf{Supongamos} que existe un modelo $\m$ y una valuación $v$ tal que $\m \vDash \Gamma[v]$. Como $\m$ satisface a todas las fórmulas de $\Gamma$, en particular satisface a $\varphi$. Luego $Q$ vale sólo para finitos elementos del modelo. Sea $k\in \N$ tal que $Q$ solo vale para ciertos elementos $e_1,\dots,e_k$. Es decir, para todo $e \in M, e \neq e_1,\dots, e_k$, no vale $Q(e)$.  	
  	
  	 Pero también tenemos que $\p_{k+1} \in \Gamma$; luego existen $\hat{e_1},\dots,\hat{e_n}\in M$, todos los elementos distintos entre sí, tal que vale $Q(\hat{e_1}),\dots, Q(\hat{e_{k+1}})$. Esto es \textbf{absurdo}\footnote{Si quieren, por el principio del palomar.}, con lo que $\Gamma$ es insatisfacible.
  	 
  	 ¿Terminó el ejercicio? Para nada. Lo único que probamos es que $\Gamma$ es insatisfacible.
  	 
  	 \item Veamos que $\Gamma$ es satisfacible. Por el teorema de compacidad vale que si todo $\Gamma_0 \subseteq \Gamma$, con $\Gamma_0$ finito, $\Gamma_0$ es satisfacible, entonces vale que $\Gamma$ es satisfacible.
  	 
  	 Sea $\Gamma_0 \subseteq \Gamma$ un subconjunto finito.
  	  Definimos $r := \max\big{(}\key{i : \p_i \in \Gamma_0} \cup \key{1}\big{)} < \infty$. Agregamos $\key{1}$ porque $\Gamma_0$ podría no tener a ninguna fórmula $\p_i$, y el conjunto al que le tomamos máximo debe ser no vacío.
  	  
  	  Definimos el siguiente modelo $\mathcal{M}$:
  	  
  	  Como dominio tomamos $M = \N$, y como relación $Q$, $Q^{\mathcal{M}}(n) = n \leq r$. En primer lugar notemos que si $\varphi \in \Gamma_0$, entonces $\varphi$ es verdadera, porque $Q$ solamente vale para $1,\dots,r$, que son finitos.
  	  
  	  Si existe $\p_n \in \Gamma_0$, tenemos que ver que existen $e_1,\dots,e_n \in M$, todos distintos, tal que vale $Q(e_1),\dots,Q(e_n)$. Tomando, para cada $1\leq i \leq n, e_i = i$, obtenemos lo deseado.
  	  
  Como $\Gamma_0$ es satisfacible, y $\Gamma_0$ era un subconjunto finito arbitrario de $\Gamma$, entonces, por compacidad, tenemos que $\Gamma$ es satisfacible.
  \end{enumerate}
  
  Hemos demostrado que $\Gamma$ es insatisfacible y satisfacible, lo cual es un absurdo. El absurdo, como dijimos antes, provino de suponer que existía tal $\varphi$. Luego la propiedad no es expresable en primer orden.\footnote{Así que no nos gastemos en intentar expresarla.}
\end{enumerate}

Con esto terminamos el ejercicio.
\end{solution}



\begin{xca}Sea $\l$ un lenguaje de primer orden con igualdad y símbolo de predicado unario $Q$. Decidir si es expresable la propiedad \say{El predicado unario $Q$ vale para \emph{infinitos} elementos del modelo.}
\end{xca}

\begin{solution}
Notemos que acá negar incrementalmente la condición \say{Valer para infinitos} está difícil. Tendríamos que crear infinitas fórmulas tal que la conjunción de todas fuerce a que el predicado valga solamente para \emph{finitos}.

Lo que sí podemos hacer es reducirla al Ejercicio 1. Supongamos que la propiedad fuera expresable vía una fórmula $\psi$. Entonces definiendo $\varphi = \neg \psi$ podríamos expresar (¡convencerse!) la negación de la propiedad, i.e., \say{El predicado unario $Q$ solo vale para \emph{finitos} elementos del modelo}, lo cual es un absurdo ya que vimos que una propiedad así no era expresable en el Ejercicio 1.
\end{solution}


\begin{xca}
Sea $\l$ un lenguaje de primer orden con igualdad y un símbolo de predicado $R$. Decidir si es expresable la propiedad \say{$R$ \textbf{no} vale solamente para finitos elementos del modelo.}.
\end{xca}
\begin{solution}
Supongamos que la propiedad fuera expresable, y definamos a $Q$ como $\neg R$. La propiedad \say{$R$ \text{no} vale solamente para finitos elementos del modelo} es equivalente a \say{$Q$ vale solamente para finitos elementos del modelo}. Hemos llegado a un absurdo, ya que en el Ejercicio 1, con $Q$ arbitrario, habíamos probado que esta propiedad no era expresable.
\end{solution}

\begin{xca}
Sea $\l$ un lenguaje de primer orden con igualdad\footnote{Para variar.} e infinitos símbolos de constante $c_1,c_2,\dots$ indexados por $\N$. ¿Es expresable la propiedad \say{$\m$ le asigna a cada constante un valor distinto}?
\end{xca}
\begin{solution}
Negar incrementalmente esto está difícil. Sin embargo, si esta propiedad fuera expresable entonces también sería expresable su negación, i.e., \say{$\m$ le asigna a dos constantes el mismo valor}

Lleguemos a un absurdo con \emph{esa} propiedad.

\begin{enumerate}
	\item Negamos incrementalmente la propiedad. Para cada $k\geq 2$, definimos:

	$\psi_k : \neg todosDistintos(c_1,\dots,c_k)$ 
	
	\item Definimos $\Gamma = \key{\psi_i : i \in \N} \cup \key{\varphi}$
	
	\item Ver que el conjunto $\Gamma$ es insatisfacible queda de ejercicio.
	
	\item Para ver que es satisfacible, por compacidad basta ver que todo subconjunto finito $\Gamma_0 \subseteq \Gamma$ es satisfacible. Sea $\Gamma_0$ un subconjunto finito y arbitrario de $\Gamma$.
	Nuevamente definimos $r = \max\key{i : \psi_i \in \Gamma_0} \cup \key{1}< \infty$.
	
	Podemos dar un modelo $\m$ con dominio los naturales, y:
	
	 $c_i^\m = \begin{cases}i &\text{ si }i \leq r
	 \\42 &\text{ si }i > r\end{cases}$
	 
	 Como antes, en primer lugar notemos que si $\varphi \in \Gamma_0$, entonces $\varphi$ es verdadera, porque $c_{r+1}^\m = c_{r+2}^\m$.
  	  
  	  Si existe $\p_n \in \Gamma_0$, tenemos que ver que existen $e_1,\dots,e_n \in M$, todos distintos, tal que vale $Q(e_1),\dots,Q(e_n)$. Tomando, para cada $1\leq i \leq n, e_i = i$, tenemos lo deseado.
  	  
  Como $\Gamma_0$ es satisfacible, y $\Gamma_0$ era un subconjunto finito arbitrario de $\Gamma$, entonces, por compacidad, tenemos que $\Gamma$ es satisfacible.

\end{enumerate}

Al demostrar que $\Gamma$ es tanto satisfacible como insatisfacible, hemos llegado a un absurdo. El absurdo provino de suponer que \say{$\m$ le asigna a cada constante un valor distinto} era una propiedad expresable en primer orden.
\end{solution}

Últimos comentarios: la resolución de este tipo de ejercicios es bastante metódica. Sin embargo, requieren algo creatividad:

\begin{itemize}
	\item ¿Será expresable la propiedad? ¡A veces sí, ojo! Ver, por ejemplo, el ejercicio 3b de este parcial \url{goo.gl/Jg9ePa}.
	\item Además, si la propiedad \emph{no} es expresable, ¿con qué trabajar? ¿Con ella, como el ejercicio 1, o con su negación, como en el ejercicio 4?
	\item Si uno hace la negación incremental mal se puede encontrar con que el conjunto $\Gamma$ no es satisfacible.
	\item ¡Para dar el modelo hay que pensar un poco!
\end{itemize}

Hagan los ejercicios de la práctica para agarrarle la mano.

\end{document}