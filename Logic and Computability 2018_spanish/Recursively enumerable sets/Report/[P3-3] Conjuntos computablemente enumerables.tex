\documentclass[10pt,a4paper]{article}
\usepackage[utf8]{inputenc}
\usepackage{amsmath}
\usepackage{amsfonts}
\usepackage{amssymb}

\usepackage[utf8]{inputenc}
\usepackage[spanish]{babel}
\usepackage{amsmath}
\usepackage{amsfonts}
\usepackage{amssymb}
\usepackage{dirtytalk}
 \usepackage{relsize}
\usepackage{mathtools}
\author{Guillermo (Billy) Mosse}
\title{Conjuntos c.e.}

\usepackage{thmtools}
\declaretheorem[name=Teorema]{theorem}
\declaretheorem[name=Lema, sibling=theorem]{lemma}
\declaretheorem[name=Definici\'on, sibling=theorem]{definition}
\declaretheorem[name=Chiste, sibling=theorem]{joke}
\declaretheorem[name=Intento de definici\'on, sibling=theorem]{trydefinition}
\declaretheorem[name=Proposici\'on, sibling=theorem]{proposition}
\declaretheorem[name=Observaci\'on, sibling=theorem]{observation}
\declaretheorem[name=Corolario, sibling=theorem]{corollary}
\declaretheorem[name=Ejemplo, style=remark, sibling=theorem]{example}
\declaretheorem[name=Ejemplos, style=remark, sibling=theorem]{examples}
\declaretheorem[name=Advertencia, sibling=theorem]{caution}
\declaretheorem[name=Nota de color, style=remark, sibling=theorem]{remark}
\def\key#1{\{#1\}}
\def\N{\mathbb{N}}
\def\bs{\mathlarger{\mathlarger{\square}}}


\usepackage[pdftex,
            pdfauthor={Guillermo Mosse},
            pdftitle={Conjuntos computablemente enumerables},
            pdfsubject={Conjuntos computablemente enumerables (c.e.), reducciones},
            pdfkeywords={reducción m, c.e.},
            pdfproducer={Latex with hyperref},
            pdfcreator={pdflatex}]{hyperref}



\begin{document}


\title{Conjuntos computablemente enumerables}
\author{Guillermo Mosse}

%TODO agregar la carátula y hacer que empiece en cero.
\maketitle
%\setcounter{page}{1}


Sugerencia: leer la clase práctica \url{https://campus.exactas.uba.ar/pluginfile.php/90823/mod_folder/content/0/conjuntos_ce_y_coce.pdf?forcedownload=1} (2do Cuatrimestre, \say{conjuntos ce y co-ce}). Especialmente el ejercicio 2.

\begin{definition}
Se dice que un conjunto $A \subseteq \N$ es c.e. si existe una función $g :\N \to \N$ parcial computable tal que:
$$A = \key{x: g(x) \downarrow } = dom\ g$$
\end{definition}

\begin{definition}
Se dice que un conjunto $A \subseteq \N$ es co-c.e. si su complemento, $\bar{A}$, es c.e.
\end{definition}

\begin{theorem}
Sea $A \subseteq \N$. Son equivalentes:
\begin{itemize}
	\item $A$ es c.e.
	\item $A$ es el rango (o sea, la imagen) de una función p.r.
	\item $A$ es el rango de una función computable.
	\item $A$ es el rango de una función parcial computable.
\end{itemize}
\end{theorem}

	\textbf{Ejercicio 1}: ¿Es todo conjunto finito c.e.? ¿Y todo conjunto co-finito?\\
	
	\textbf{Solución}: Dado un conjunto finito $C = \key{x_1,\dots,x_n}$, podemos armar un programa $P$ que devuelva $0$ en esos elementos, y sino se cuelgue. Por ejemplo:
	$$IF\ X= x_1\ GOTO\ [E]$$
	$$\dots$$
	$$IF\ X= x_n\ GOTO\ [E]$$
	$$[A]\ \ GOTO\ [A]$$
	

Sea $f$ la función que $P$ computa. Es fácil ver que $C = dom\ f$, así que $C$ es el dominio de una función computable, y por lo tanto es c.e.

Otra manera de responder la pregunta es probando que $C$ es computable (sugerencia: armar un programa con $n$ IFs, como el de arriba) y entonces, por la slide 120 de la teórica, es c.e.

Para la otra pregunta, si $C$ es co-finito, es decir, si $\bar{C}$ es finito, entonces $\bar{C}$ es computable. Es decir, existe $g$ computable tal que $g(x) = \begin{cases} 1$ si $x \in \bar{C}\\0$ si no$\end{cases}$

y por lo tanto $C$ también es computable, vía $h(x) = \neg\ g(x)$, es decir,

$h(x) = \begin{cases} 0$ si vale $g(x)\\1$ si no$\end{cases}$

Como $C$ es computable, de nuevo tenemos que es c.e.


Recién usamos que si un conjunto es computable entonces su complemento también. ¿Vale lo mismo para c.e.?

\begin{observation}(Teórica) (Importante)
	$C$ c.e. y co-c.e. $\Leftrightarrow C$ computable
\end{observation}

(Está bueno pensar cómo probar esto: ¿cómo armarse un programa que se fije si un elemento está en $C$ sabiendo que tanto $C$ como $\bar{C}$ son el rango de funciones computables?)

Si valiera que si $C$ es c.e. entonces su complemento es c.e., entonces todo conjunto c.e. sería computable por la observación de arriba. Pero $K:=\key{x : \Phi_x(x) \downarrow}$ es c.e., y no computable, así que:

\begin{observation}
	En general \textbf{no vale} $C$ c.e. $\Rightarrow \bar{C}$ c.e.
\end{observation}

La siguiente definición es nueva:

\begin{definition}
		Sea $A,B \subset \N$. Se dice que $A$ es reducible a $B$, y se nota  $A\leq B$, si existe una función $f$ \textbf{computable} (total) tal que $\forall\ x \in \N, x \in A$ sii $f(x) \in B$. 		Además decimos que $A \equiv B$ si $A\leq B$ y $B \leq A$.
\end{definition}

Pensemos que un conjunto puede ser computable. Si no es computable, al menos puede ser c.e. O también podría no ser ni siquiera c.e., como $TOT$. Si $A \leq B$, entonces $B$ \say{acota} la \say{dificultad} de $A$.

Esto puede resultar intuitivo por la siguiente observación: supongamos que $B$ es computable. ¿Cómo decidir si $x \in A$? Como $B$ es computable, podemos simplemente chequear si $f(x) \in B$. Computar $A$ se reduce a computar $B$. Pero esta cota de dificultad también se da en otros sentidos:\\
	
	\textbf{Ejercicio 2}: Probar que si $A \leq B$, entonces $B$ c.e implica $A$ c.e.\\
	
\textbf{Solución}: Una posibilidad es construirnos una función $h$ parcial computable tal que $A = dom\ h$. $B$ es c.e. así que existe $g$ computable tal que $Dom\ g = B$. Además, como $A\leq B$, existe $f$ computable tal que $\forall\ x \in \N, x \in A$ sii $f(x) \in B$. ¿Cómo podemos armarnos la función $h$ que necesitamos? Queremos que se defina solamente en los elementos de $A$. Tenemos que $f$ manda elementos de $A$ (y \textbf{sólo} los de $A$) a $B$, y $g$ sólo se define en los elementos de $B$. Afirmamos que $Dom\ g\circ f = A$. En efecto, solo hay que seguir los si y solo si que cumple cada función:

$$x \in A\ sii\ f(x) \in B\ sii\ g(f(x)) \downarrow$$
	
	\textbf{Ejercicio 3} (tarea, y muy útil): Probar lo mismo, cambiando \say{c.e.} por:
	\begin{enumerate}
		\item computable (formalizar lo discutido antes)
		\item co-ce
	\end{enumerate}


	\textbf{Ejercicio 4}: completar con $\leq, \geq$ ó $\equiv$\\
	
	
	(El primer ítem fue agregado después de la clase)\\



\textbf{a)}
	\begin{equation*}
	\key{2\cdot n : n \in \N}\ \raisebox{-1pt}{$\bs$} \ \key{4\cdot n : n \in \N}
	\end{equation*}
	
	\textbf{Solución}: Vale $\equiv$. Llamemos $C_1$ al primer conjunto, y $C_2$ al segundo.\\
	
	$\leq$: Sea $f(x) := 2x$, claramente computable. Tenemos que ver que $\forall\ x\in \N,\\ x \in C_1 \Leftrightarrow f(x) \in C_2$. Pero esto es fácil, porque $x \in C_1 \Leftrightarrow \exists\ n\in \N$ tal que $x = 2n \Leftrightarrow \exists\ n\in \N$ tal que $f(x) = 2\cdot 2n = 4n \Leftrightarrow x \in C_2$, por definición de $C_2$.\\
	
	$\geq$: para esta hay que tener un poco más de cuidado, porque dividir por $2$ no funciona (¿por qué?). Pero aprovechando todo lo que hicimos en la práctica 1, podemos definir por ejemplo:
	
	$g(x) = \begin{cases} 2$ si $x \in C_2\\1$ si no$\end{cases}$
	
	porque en la definición de reducción no se pide que la función sea \textit{inyectiva}.
	
	Claramente $x\in C_2 \Leftrightarrow g(x) \in C_1$, porque $\forall\ x\in C_2, g(x) = 2 \in C_1$, y $\forall\ x\not\in C_2, g(x) = 1\not\in C_1$. \\

\textbf{b)}
	\begin{equation*}
	\raisebox{-1pt}{$\N$}\ \raisebox{-2pt}{$\bs$} \ \raisebox{-1pt}{$\key{2\cdot n : n \in \N}$}
	\end{equation*}	
	
	\textbf{Solución}: Llamemos $C_3$ al conjunto de la derecha. Para ver que vale $\leq$ podemos hacer algo parecido a lo anterior, y usar la función $f(x) = 2x$. Trivialmente, todo $x\in \N$ cumple que $x\in \N$, así que lo que hay que ver es que $\forall\ x\in \N, f(x) \in C_3$, pero esto es obvio porque el conjunto es el de \textit{los pares}, y la función \textit{multiplica por 2}.
	
	No vale $\geq$, y esto es triqui; en realidad $\N$ es un caso borde, medio patológico de este orden. Para probar que $\N \not\geq \key{2\cdot n : n \in \N}$, hay que ver que $\forall\ g$ función computable, NO vale $x \in C_3 \Leftrightarrow g(x) \in \N$. Esto es más fácil de lo que parece. Sea $x=1$. Tenemos que $x \not\in C_3$, pero siempre, cualquiera sea $g$, $g(x)\in \N$, así que no se cumple el si y solo sí (más específicamente, no se cumple $\Leftarrow$, por contrarrecíproco).\\

\textbf{c)}
	\begin{equation*}
	K\ \bs\ \key{x : \phi_x(x) \downarrow \land \phi_x(x) = 42^{42}}
	\end{equation*}
	
	\underline{Idea}: si un programa está definido en $x$ puedo cambiar su valor a otro número.
	
	Llamemos $K_2$ al conjunto de la derecha. Veamos que vale $\leq$. Queremos una función $f$ computable tal que $x\in K \Leftrightarrow f(x) \in K_2$, es decir, queremos que $\phi_x(x) \downarrow \Leftrightarrow\phi_{f(x)}(f(x)) =42$.

Citando a María Emilia Descotte, \say{estamos buscando entonces una f computable tal que cuando la uso como número de programa le pase algo particular, eso suena a teorema del parámetro.} 

Sea \[g(x, y) = \begin{cases}
42^{42} & \text{ si } \phi_x(x) \downarrow\\
\uparrow & \text{ si no}
\end{cases}\]

parcial computable. La variable $y$ no hace nada, está simplemente para que cuando usemos el Teorema del Parámetro no obtengamos una función 0-aria (lo cuál sería un \say{abuso}, como hicimos en clase).

Sea $e$ tal que $\Phi^{(2)}_e(x, y) = g(x, y)$.

Por el teorema del parámetro, existe $S^1_1$ p.r. que fija la variable $x$, es decir, tal que $\Phi^{(2)}_e(x,y) = \Phi^{(1)}_{S^1_1(x,e)}(y)$.

Tenemos que $x \in K \Leftrightarrow S^1_1(x,e) = 42^{42} \Leftrightarrow S^1_1(x,e) \in \key{x : \phi_x(x) \downarrow\ \land \phi_x(x) = 42^{42}}= K_2$. La función que nos sirve, entonces, es $g(x) = S^1_1(x,e)$. (Notar que $e$ es un número fijo.)

Ver si vale $\geq$ queda de ejercicio.\\

\textbf{d)} Algo un poquito más difícil de ver es lo siguiente, y lo dejamos para pensar:
	\begin{equation*}
	K\ \bs\ TOT
	\end{equation*}
	
Para ver que \textbf{no} vale $\geq$, supongamos que vale. Entonces, por el \textbf{Ejercicio 3}, como $K$ es c.e., tendríamos que $TOT$ es c.e. Pero en la teórica vimos que esto no es cierto, absurdo! Así que $K\not\geq TOT$. \\

	

Más para pensar: ¿Siempre se pueden comparar dos conjuntos? ¿Es \textit{total} el orden? 
	
	Un posible contraejemplo es $\emptyset$ y $\N$. Pero piensen uno menos trivial! Sugerencia: usen que c.e. y co-c.e. implica computable, y que si $A \leq B$ entonces $B$ c.e. implica $A$ c.e. y $B$ co-c.e. implica $B$ co-c.e. \\
	
	\textbf{Ejercicio 5}: sea $A =\key{x : \phi_x(x) \downarrow \land \phi_x(x) = 42^{42}}$, como arriba.
	¿Es $A$ c.e.? ¿Es $A$ co-c.e.?
	
	Tenemos que $K \leq A$, así que $A$ no puede ser co-c.e., porque eso implicaría que $K$ es co-c.e., y por lo tanto computable.
	
	Además es claro que $A$ es c.e., porque es el dominio de la función:
	
	$g(x) = \begin{cases} 0\text{ si } \phi_x(x) \downarrow \land\ \phi_x(x) = 42^{42}\\ \uparrow\text{ en caso contrario} \end{cases}$, que es parcial computable. \\
	
	\textbf{Ejercicio 6}. Decidir $V$ ó $F$ y justificar.

\begin{itemize}
	\item Sea $A$ computable. Entonces existe $f$ tal que el conjunto $f(A)$ es no computable.

Falso, si $A = \emptyset$ entonces $f(A) = \emptyset$ y por lo tanto es computable.

	\item Sea $\emptyset \neq A$ computable y no vacío. Entonces existe $f$ tal que $f(A)$ es no computable.
	
	Parecido. Falso. Si $A$ es finito entonces $f(A)$ es finito y por lo tanto computable.
	
	\item Sea $A$ computable e infinito. Entonces existe $f$ tal que $f(A)$ es no computable.
	
	No le estamos pidiendo ninguna hipótesis a $f$. Si $A$ es infinito entonces lo podemos biyectar con $K$. Numeremos a los elementos de los conjuntos. Por ejemplo, escribiéndolos de menor a mayor, tenemos que $A = \key{a_1, a_2, \dots}$ y $K = \key{k_1, k_2, \dots}$. Entonces podemos definirnos:
	
	 $f(x) = \begin{cases} k_i$ si $x = a_i\\ 42^{42}$ en caso contrario$\end{cases}$
	 
	 Tal vez esto confunde porque estamos acostumbrados a funciones definidas por alguna fórmula, como $2x, 2^x, 3x+1$, o por cómputos de programas, pero una función cualquiera no es nada más que un objeto matemático que le asigna un valor de la imagen ($\N$) a todo valor del dominio (también $\N$). \footnote{La definición formal que se suele ver en Álgebra 1 es con producto cartesiano...pero no importa.}
	
	\item Sea $A$ computable e infinito. Entonces existe $f$ computable total tal que $f(A)$ no es computable.
	
	Este es un poco más triqui. Pensemos un poco...si $A = \N$ entonces la conjetura es cierta, porque cualquier conjunto c.e. es la imagen de una función computable. En otras palabras, $\exists f:\N \to \N$ computable tal que el conjunto c.e. es igual a $f(\N)$. Por ejemplo, para el conjunto $K$, que es c.e., pero no computable, existe una función $f$ de esta pinta (que \say{enumera} a $K$). Entonces $f(\N) = K$ y estamos.
	
	¿Y si $A$ no es $\N$? Bueno, si primero pudiéramos mandar $A$ a $\N$ vía una función computable, despueś mandamos $\N$ a $K$ y estamos. Es decir, queremos esto:
	
	$$A \overset{h}{\longrightarrow} \N \overset{f}{\longrightarrow} K$$
	
	Como $A$ es computable, $\forall\ x\in \N$ podemos decidir si $x\in A$ o no. Esto hace las cosas muy fáciles. $h$ puede ser la función que computa el siguiente programa:
	
	
	\begin{align*}
	&WHILE (Z \leq X)\\
	&\ \ \ \ IF\ A(Z)\ THEN\ Y = Y+1\\
	&\ \ \ \ Z = Z + 1\\
	&Y=Y \dot{-} 1
	\end{align*}

El programa no hace más que contar la cantidad de elementos que están en $A$ que sean menores o iguales a $x$, y luego restar 1. (Ese restar uno está por el detalle de que queremos que algún elemento vaya al 0).

Dicho más matematicosamente\footnote{es más gracioso que decir \say{matemáticamente}}, para cada entrada $x_1$ devolvemos $|\key{z : z \leq x \land z \in A}| \dot{-} 1$.  Notemos que la función, expresada así, resulta claramente computable porque es composición de funciones computables.


Si nuevamente escribimos a $A$ de manera ordenada, $A = \key{a_1, a_2, \dots}$, cuando el programa reciba como entrada a $a_1$ va a devolver $0$, y en general va a mandar $a_i \mapsto i-1$.

Sea $f$ la función computada por este programa. Como dijimos, $f(A) = \N$. Repitiéndonos un poco, para concluir el ejercicio basta agarrar un conjunto c.e. no computable cualquiera, como $K$. Sea $g$ una función enumeradora de $K$. Entonces $g\circ f(A) = K$, y estamos.
	
\end{itemize}


\end{document}