\documentclass[11pt]{beamer}
\usetheme{Copenhagen}
\usepackage[utf8]{inputenc}
\usepackage[spanish]{babel}
\usepackage{amsmath}
\usepackage{amsfonts}
\usepackage{amssymb}
\usepackage{dirtytalk}
 \usepackage{relsize}
\uselanguage{spanish}
\languagepath{spanish}
\deftranslation[to=spanish]{Theorem}{Teorema}
\deftranslation[to=spanish]{theorem}{teorema}
\deftranslation[to=spanish]{Definition}{Definición}
\deftranslation[to=spanish]{definition}{definición}
\usepackage{mathtools}
\author{Guillermo (Billy) Mosse}
\title{Conjuntos c.e.}
\setbeamercovered{transparent} 
\setbeamertemplate{navigation symbols}{} 
%\logo{} 
\institute{FCEyN, UBA} 
\date
%\subject{} 

\def\key#1{\{#1\}}
\def\N{\mathbb{N}}
\def\bs{\mathlarger{\mathlarger{\mathlarger{\square}}}}

\usepackage{etoolbox}

\setbeamercolor{block title}{use=structure,fg=structure.fg,bg=structure.fg!20!bg}
\setbeamercolor{block body}{parent=normal text,use=block title,bg=block title.bg!50!bg}

\setbeamercolor{block title exercise}{use=example text,fg=example text.fg,bg=example text.fg!20!bg}
\setbeamercolor{block body exercise}{parent=normal text,use=block title example,bg=block title example.bg!50!bg}


\BeforeBeginEnvironment{exercise}{
    \setbeamercolor{block title}{fg=black,bg=orange!50!white}
    \setbeamercolor{block body}{fg=black, bg=orange!30!white}
}
\AfterEndEnvironment{exercise}{
 \setbeamercolor{block title}{use=structure,fg=structure.fg,bg=structure.fg!20!bg}
 \setbeamercolor{block body}{parent=normal text,use=block title,bg=block title.bg!50!bg, fg=black}
}

\newtheorem{exercise}{Ejercicio}
\setbeamercovered{transparent}
\begin{document}
\setbeamercovered{invisible}


\begin{frame}
\titlepage
\end{frame}

%\begin{frame}
%\tableofcontents
%\end{frame}

\begin{frame}{Introducción}
\begin{definition}
Se dice que un conjunto $A$ es c.e. si existe una función $g :\N \to \N$ parcial computable tal que:
$$A = \key{x: g(x) \downarrow } = dom\ g$$
\end{definition} \pause

\begin{theorem}
Sea $A \subset \N$. Son equivalentes:
\begin{itemize}
	\item $A$ es c.e.
	\item $A$ es el rango de una función p.r.
	\item $A$ es el rango de una función computable.
	\item $A$ es el rango de una función parcial computable.
\end{itemize}
\end{theorem} \pause

\begin{exercise}
	¿Es todo conjunto finito c.e.? ¿Y todo conjunto co-finito?
\end{exercise}

\end{frame}

\begin{frame}{Una definición más}
	\begin{definition}
		Sea $A,B \subset \N$. Se dice que $A \leq B$ si existe una función $f$ \textbf{computable} tal que $\forall\ x \in \N, x \in A$ sii $f(x) \in B$. \\
		
		Decimos que $A \equiv B$ si $A\leq B$ y $B \leq A$.
	\end{definition} \pause
	
	\begin{exercise}	
	Probar que si $A \leq B$, entonces $B$ c.e implica $A$ c.e.
	\end{exercise} \pause

	\begin{exercise}{(Tarea)}
	Probar lo mismo, cambiando \say{c.e.} por:
	\begin{enumerate}
		\item computable
		\item co-ce	(co-c.e. es que el complemento sea c.e.)
	\end{enumerate}
	\end{exercise}	
\end{frame}


\begin{frame}{Como en la primaria}
	\begin{exercise}
	Completar con $\leq, \geq$ ó $\equiv$
	
	\begin{equation*}
	\raisebox{-1pt}{$\N$}\ \raisebox{-2pt}{$\bs$}\ \key{2\cdot n : n \in \N}
	\end{equation*}\pause
	
	\begin{equation*}
	\raisebox{-1pt}{$K$}\ \raisebox{-2pt}{$\bs$}\ \key{x : \phi_x(x) \downarrow \land \phi_x(x) = 42^{42}}
	\end{equation*}\pause
	
	\begin{equation*}
	K\ \raisebox{-2pt}{$\bs$}\ TOT\text{ (tarea)}
	\end{equation*}\pause
	\end{exercise}
	
	¿Siempre se pueden comparar dos conjuntos? ¿Es \textit{total} el orden? \pause
\end{frame}

\begin{frame}{}
	(Dejamos de nuevo la definición de $\leq$)
	
	\begin{definition}
		Sea $A,B \subset \N$. Se dice que $A \leq B$ si existe una función $f$ \textbf{computable} tal que $\forall\ x \in \N, x \in A$ sii $f(x) \in B$. \\
		
	Decimos que $A \equiv B$ si $A\leq B$ y $B \leq A$.
	\end{definition}
	
	Además teníamos que:
	
	\begin{equation*}
	K\ \leq \key{x : \phi_x(x) \downarrow \land \phi_x(x) = 42^{42}}
	\end{equation*}


	\begin{exercise}
	Ejercicio 5: sea $A =\key{x : \phi_x(x) \downarrow \land \phi_x(x) = 42^42}$, como arriba.
	¿Es $A$ c.e.? ¿Es $A$ co-c.e.?
	\end{exercise}	
\end{frame}

\begin{frame}{Ping Pong}
Ejercicio 6. Decidir $V$ ó $F$ y justificar.

\begin{itemize}
	\item Sea $A$ computable. Entonces existe $f$ tal que $f(A)$ es no computable. \pause
	\item Sea $\emptyset \neq A$ computable y no vacío. Entonces existe $f$ tal que $f(A)$ es no computable.\pause
	\item Sea $A$ computable e infinito. Entonces existe $f$ tal que $f(A)$ es no computable.\pause
	\item Sea $A$ computable e infinito. Entonces existe $f$ computable total tal que $f(A)$ no es computable. \pause
\end{itemize}
	
\end{frame}

\end{document}