%\documentclass[11pt,a4paper]{report}


% The Beamer class comes with a number of default slide themes
% which change the colors and layouts of slides. Below this is a list
% of all the themes, uncomment each in turn to see what they look like.

%\usetheme{default}
%\usetheme{AnnArbor}
%\usetheme{Antibes}
%\usetheme{Bergen}
%\usetheme{Berkeley}
%\usetheme{Berlin}
%\usetheme{Boadilla}
%\usetheme{CambridgeUS}
%\usetheme{Copenhagen}
%\usetheme{Darmstadt}
%\usetheme{Dresden}
%\usetheme{Frankfurt}
%\usetheme{Goettingen}
%\usetheme{Hannover}
%\usetheme{Ilmenau}
%\usetheme{JuanLesPins}
%\usetheme{Luebeck}
%\usetheme{Madrid}
%\usetheme{Malmoe}
%\usetheme{Marburg}
%\usetheme{Montpellier}
%\usetheme{PaloAlto}
%\usetheme{Pittsburgh}
%\usetheme{Rochester}
%\usetheme{Singapore}
%\usetheme{Szeged}
%\usetheme{Warsaw}

\documentclass[10pt]{beamer}
\usepackage{float}
\usepackage[nodisplayskipstretch]{setspace}
\usepackage{pdfpages}
\usepackage{tikz}    
\usetheme{metropolis}
\usepackage{appendixnumberbeamer}
\usepackage[normalem]{ulem}
\usepackage{eurosym}
\usepackage{booktabs}
\usepackage[scale=2]{ccicons}
\usepackage[utf8]{inputenc}
\usepackage{soul}
\usepackage{mathabx}
\usepackage{graphicx}
\usepackage{pgfplots}
\usepgfplotslibrary{dateplot}
 \usepackage{relsize}
\usepackage{xspace}
\usepackage{caption}

\usepackage{graphicx}
\usepackage{graphicx}

\usepackage{hyperref}
\hypersetup{
    colorlinks=true,
    linkcolor=blue,
    filecolor=magenta,      
    urlcolor=blue,
}
 
\urlstyle{same}
\newcommand{\themename}{\textbf{\textsc{metropolis}}\xspace}


% As well as themes, the Beamer class has a number of color themes
% for any slide theme. Uncomment each of these in turn to see how it
% changes the colors of your current slide theme.

%\usecolortheme{albatross}
%\usecolortheme{beaver}
%\usecolortheme{beetle}
%\usecolortheme{crane}
%\usecolortheme{dolphin}
%\usecolortheme{dove}
%\usecolortheme{fly}
%\usecolortheme{lily}
%\usecolortheme{orchid}
%\usecolortheme{rose}
%\usecolortheme{seagull}
%\usecolortheme{seahorse}
%\usecolortheme{whale}
%\usecolortheme{wolverine}

%\setbeamertemplate{footline} % To remove the footer line in all slides uncomment this line
%\setbeamertemplate{footline}[page number] % To replace the footer line in all slides with a simple slide count uncomment this line

%\setbeamertemplate{navigation symbols}{} % To remove the navigation symbols from the bottom of all slides uncomment this line
\usepackage{tikz-cd}

\usepackage[utf8]{inputenc}
\usepackage{graphicx}

\usepackage{amsmath,amsthm,amssymb,latexsym,amsfonts}
\usepackage{tikz}
\newcommand*\circled[1]{\tikz[baseline=(char.base)]{
   \node[shape=circle,color=red,draw,inner sep=1pt] (char) {#1};}}


\setlength{\parindent}{15pt}
\usepackage{subfig}
\usepackage{hyperref}
\usepackage{graphicx} % Allows including images
\usepackage{booktabs} % Allows the use of \toprule, \midrule and \bottomrule in tables

\usepackage{bm}

\newtheorem{caution}[theorem]{¡Cuidado!}

\def\Q{\mathbb{Q}}
\def\R{\mathbb{R}}
\def\C{\mathbb{C}}
\def\N{\mathbb{N}}
\def\Z{\mathbb{Z}}
\def\S{\mathcal{S}}
\def\H{\mathcal{H}}
%\def\sde{\underset{\theta}{\times}}
\def\sde#1{\underset{#1}{\times}}

\def\G{Sea $G$ un grupo}
\def\GG{Sean\ $G, G'$\ grupos}
\def\GS{Sea\ $G$ un grupo y $S\in\S(G)$}
\def\ac{Sea $\cdot: G\times X \rightarrow X$ una acción}

\def\dcup{\sqcup}

\def\gen#1{<#1>}
\def\genn#1{<\bar{#1}>}
\def\key#1{\{#1\}}

%tetration
\def\inv#1#2{(#2,#1)}

\def\ii{\textbf{i}}
\def\jj{\textbf{j}}
\def\kk{\textbf{k}}

\def\div{por el algoritmo de división, existen $q,r$ tal que\xspace}
\def\Div{Por el algoritmo de división, existen $q,r$ tal que\xspace}
\def\m{^{-1}}

%Cambiar por un triangulito
\def\n{\Delta}

\def\p{Sea $p$ un número primo}

\def\I{[a,b]} 


\usepackage[utf8]{inputenc}
\usepackage{graphicx}
\graphicspath{ {images/} }

\usepackage[export]{adjustbox}
\usepackage{hyperref}
\usepackage{graphicx} % Allows including images
\usepackage{booktabs} % Allows the use of \toprule, \midrule and \bottomrule in tables

\usepackage{etoolbox}

\addtobeamertemplate{proof begin}{%
	\setbeamercolor{block title}{fg=black,bg=red!50!white}
	\setbeamercolor{block body}{fg=red, bg=red!30!white}
}{}


\BeforeBeginEnvironment{definition}{
	\setbeamercolor{block title}{fg=black,bg=green!20!gray}
	%\setbeamercolor{block body}{fg=black, bg=green!40!gray}
}

\AfterEndEnvironment{definition}{
	\setbeamercolor{block title}{fg=black,bg=green!20!gray}
	%\setbeamercolor{block body}{fg=black, bg=green!40!gray}
}

\BeforeBeginEnvironment{theorem}{
	\setbeamercolor{block title}{fg=black,bg=gray!40!white}
	%\setbeamercolor{block body}{fg=black, bg=green!40!gray}
}

\AfterEndEnvironment{theorem}{
	\setbeamercolor{block title}{fg=black,bg=gray!40!white}
	%\setbeamercolor{block body}{fg=black, bg=green!40!gray}
}



\title{Cuaterniones}


\begin{document}

\maketitle

\section{Segunda clase: cuaterniones!}


\begin{frame}{Operatoria de cuaterniones}
¿Cómo eran los cuaterniones? \pause

$$a + b \cdot \ii + c \cdot \jj + d \cdot \kk$$

\pause


\begin{itemize}
        \item En general no conmutan (esa propiedad se pierde)
		\item El producto de un número real y un cuaternión sí conmutan.
		\item La suma de dos cuaterniones es conmutativa
		\item La suma y el producto es asociativo		
\end{itemize} \pause

¿A ustedes como les aparecen cuando los usan?
	% Un cuaternión es elegir valores de a,b,c,d. Si elegimos todos cero salvo b obtenemos $i$, y si elegimos todos cero salvo $a$ obtenemos un número real
	% Y esos sí conmutan, pero no nos metamos mucho en eso.
	
	% Uno puede multiplicar usando la doble distributiva para ver cómo se llevan las letras y listo. Hagamos un ejemplo
\end{frame}

\begin{frame}{¿Y por qué no conmutan?}
    ¿Por qué en general $q_1 \cdot q_2 \neq q_2 \cdot q_1$? \pause
    
    \includegraphics[scale=0.6]{dontcommute.jpg}
    
    Porque las rotaciones no conmutan.
    
\end{frame}

\begin{frame}{¡Aun hay más!}

%Hackcito porque me da paja usar align
\begin{equation}
\begin{tikzcd}
\ \ \ \ \ \ \ \ \ \ \ \ \ Sedeniones \text{\sout{(son una bosta)}}\\
Octoniones \arrow{u} \\
Cuaterniones \arrow[u, "o_1 \cdot (o_2 \cdot o_3) \neq (o_1 \cdot o_2) \cdot o_3"'] \\
Complejos  \arrow[u, "q_1 \cdot q_2 \neq q_2 \cdot q_1"', u] \\
Reales \arrow[u, "\text{todo es lindo}"'] 
\end{tikzcd}
\end{equation}


\end{frame}


%Esto hacerlo en el pizarrón
% Notación
% Les cuento de paso que a veces las cosas se abrevian así
%$q = (q_0, \textbf{q})$, donde $q=(q_0,q_1,q_2,q_3)$ y $\textbf{q}= (q_1,q_2,q_3)$

\begin{frame}{Cuaterniones como DigiEvolución de los números complejos}
\tiny{Igual a mí me gustaba más Pokemon.} \bigskip

\normalsize
%TODO llevar las reglas para operar

% DigiEvolución es algo de un dibujito, no es un término matemático, aclaro

En $2D$, si $v,z$ son números complejos, cuando los multiplico me da $v$ rotado en el ángulo de $z$
$$w = v \cdot z$$ \pause \bigskip

En $3D$, si $v$ tiene 3 dimensiones y $q$ es un cuaternión de longitud $1$, esta cuenta me devuelve $v$ rotado 'según' el cuaternión:

$$\hat{w} = q \cdot \hat{v} \cdot q^*$$ \pause

\hbox{\hspace{8.5cm} \includegraphics[scale=0.18]{carl-sagan-dude-what.jpg}}

% Entonces hay dos cosas que queremos hacer:
% La primera es entender la fórmula
% La segunda es entender qué significa %'según' el cuaternión%
% Fíjense que aparece un cuaternión, el vector v que queremos rotar con una raya arriba, y una estrella

%PIZARRÓN escribirlo ahí para que quede

\end{frame}


\begin{frame}{Conjugación}

% Cambiemos la notación un poco, y en vez de poner a,b,c,d, pongamos q_0,q_1, q_2,q_3.
Dado $q=q_0 + q_1 \textbf{i}+ q_2 \textbf{j}+ q_3 \textbf{k}$, definimos:

$q^* = q_0 -  q_1 \textbf{i} -  q_2 \textbf{j}-  q_3\textbf{k}$ el conjugado de $q$. \bigskip


Con los complejos era parecido pero había menos letras:

\includegraphics[scale=0.5]{conjugado.png}

\iffalse
\pause \bigskip



Es lo mismo que decir lo siguiente:

Si $q = (q_0, \textbf{q}
)$, entonces $\bar{q} = (q_0, - \textbf{q}
)$

% Le pongo un menos al vector de $\R^3$
\fi

\end{frame}


\begin{frame}{De paso: longitud de un cuaternión}
    
    Longitud de un número complejo $a+b\cdot \ii: \sqrt{a^2+b^2}$ \bigskip
    
    'Longitud' de un cuaternión $a+b\cdot \ii + c\cdot \jj + d\cdot \kk: \sqrt{a^2+b^2+c^2+d^2}$ \pause
    
    Vale que si la longitud del cuaternión es $1$, entonces el inverso multiplicativo es el conjugado.
    
    ¿Se acuerdan por qué valía con complejos?
\end{frame}

\begin{frame}{'Aumentar' $v \in \R^3$ (3 dim) a $\hat{v} \in Cuat$ }

%Bueno ya sabemos qué hace la estrella
% Ahora aparentemente si tengo un vector de $\R^3$ le puedo poner una barra arriba y eso hace que pase algo

% Lo que sucede es lo siguiente: queremos 'juntar' un vector de $\R^3$ con un cuaternión de $\R^4$, pero tienen dimensiones distintas. Así que al vector de $\R^3$ lo aumentamos. ¿Cómo? Haciendo que la parte que no tiene letra, que la gente llama "parte escalar" sea 0.

Dado $v = (v_1,v_2,v_3)$, defino $\hat{v}= 0 + v_1 \cdot \textbf{i} + v_2 \cdot \textbf{j} + v_3 \cdot \textbf{k}$ \pause

Lo 'aumento' para que tenga 4 dimensiones.


%Otra manera de verlo es que $ \hat{v} =(0,v)$ (y tiene $4$ dimensiones) 
%Por eso es buena esta notación
	
\end{frame}

\begin{frame}{Volvamos a la formulita}

Si $v$ es un vector de 3 dimensiones en el espacio y $q$ es un cuaternión de longitud $1$, esta cuenta me devuelve $v$ rotado 'según' el cuaternión:

$$\hat{w} = q \cdot \hat{v} \cdot q^*$$ \pause

Pasos:

\begin{itemize}
	\item 'Aumento' al vector $v$
	\item Lo multiplico por $q$
	\item Lo multiplico por $q^*$, que es el conjugado de $q$
	\item Eso rota $v$, pero, el resultado queda 'aumentado' (o sea, la parte escalar, que no acompaña a ninguna letra, es $0$)
	
	
	% Vamos a hacer un ejemplo pero primero entendamos QUÉ rotación representa ese cuaternión
\end{itemize}
\end{frame}


\begin{frame}{Multiplicar los cuaterniones es como componer las rotaciones}

    ¿Pero en qué orden? \pause
    
    Si tengo:
    
    $$\hat{v_2} = p \cdot \hat{v_1} \cdot p^*$$
    
    y luego hago:
    
    $$\hat{v_3} = q \cdot \hat{v_2} \cdot q^*$$ \pause
    
    Esto es lo mismo que:
    
    $$\hat{v_3} = qp \cdot \hat{v_2} \cdot (qp)^*$$
    
\end{frame}
\begin{frame}

    \includegraphics[scale=0.4]{nonsense.JPG}
    
\end{frame}




\begin{frame}{Slide optativa: ¿por qué necesitamos dos números más que en $2D$?}

\large Hay dos cuentas que muestran por qué no anda agregar sólo \textbf{\ii} y \textbf{\jj}:

\normalsize

\begin{itemize}
    \item Porque necesitamos 4 letras para describir a todas las rotaciones en $3D$. \pause
    \item Si usáramos sólo $\ii$ y $\jj$ el álgebra se rompería. (Suponés que $\ii\jj=a+b\ii+c\jj$, multiplicás a izquierda por $\ii$ y llegás a algo imposible)
\end{itemize}

\iffalse Pero supongamos que anda esto de usar 3 variables  Las podríamos representar con un vector $(x,y,z)$ o con "números" de la forma $a+b\ii+c\jj$. \\

Hamilton quería que $i^2=j^2=-1$ para generalizar a los números complejos.


¿Qué pasa cuando multiplicamos $\ii$ por $\jj$? Da...algo. Digamos que deben existir $a,b,c\in\R$ tal que $\ii\jj=a+b\ii+c\jj$. Pero multiplicando a cada lado por $\ii$ obtenemos $-j = a\ii-b+c\ii\jj$. Pero...ya sabemos cuánto vale $\ii\jj$. Reemplazando, obtenemos lo siguiente:

$-\jj=a\ii-b+c(a+b\ii+c\jj)$, o sea, $0=(ac-b)+\ii(a+bc)+\jj(c^2+1)$. Y esto implica que $c^2+1 = 0$, lo cuál no puede ser.
\fi
\end{frame}

\begin{frame}{Los cuaterniones como rotaciones (con ejemplo)}
%Si $u\in\R^3, r\in\R$, entonces $q=r(cos\theta+u\ sin\theta)$ es una rotación de ángulo $2\theta$ a través de $u$.

%¿Qué significa esto?

Si tenemos el eje (el mango del pisapapas) y el ángulo con que queremos rotar, es inmediato definirse el cuaternión que describe \textit{esa rotación}: \pause

Si $N= (n_x,n_y,n_z)$ es un punto (vector) de longitud $1$ (el mango del pisapapas) y $\theta$ es un ángulo, entonces: \pause
$$q = cos(\theta/2) + sen(\theta/2)\ n_x\ \textbf{i} + sen(\theta/2)\ n_y\ \textbf{j} + sen(\theta/2)\ n_z\ \textbf{k}$$
% Es como la forma trigonométrica de los complejos
% que tenía el ángulo bien a la vista.

representa una rotación en los planos perpendiculares a $\hat{n}$ con ese ángulo. \pause

(¿con complejos era parecido?)

%TODO regla de la mano derecha

%TODO escribir ANTES que los cuaterniones tienen que tener módulo 1?

\pause Observaciones: 

\begin{itemize}
	\item Si uso $q$ ó $-q$ obtengo la misma rotación. Pero, salvo esto, por cada rotación del espacio hay un sólo cuaternión que la representa.
	\item Con $q=1 + 0\ \textbf{i} + 0\ \textbf{j} + 0\ \textbf{k}$ obtengo la rotación 'no hacer nada'. %(ver en el pizarrón)
\end{itemize}

%Una manera más compacta de escribir la fórmula de arriba es la siguiente:

$$q(\theta, \hat{n}) = (cos(\theta/2), \hat{n}\ sen(\theta/2))$$

% PIZARRÓN: escribirla para que quede y la pueda usar para explicar por qué se dobla el ángulo

%Hacer ejemplo: pasar v=(1,0,1) a (0,1,1) usando el eje (0,0,1), que tiene módulo 1, y ángulo pi/2.

% Por trigonometría, sen(pi/4)=cos(pi/4) = \frac{sqrt[2]{2}}{2}

\end{frame}

\begin{frame}{¿Por qué multiplicamos el ángulo por $2$?}

Dos razones:

\begin{itemize}
    \item como estamos multiplicando dos veces por el cuaternión necesitamos que cada uno 'tenga' la mitad del ángulo de rotación.
    \item Si \textbf{no} multiplicáramos el ángulo por dos, tendríamos el siguiente problema (dibujito)
\end{itemize}

%TODO acordarse de dibujar la esfera!

%PIZARRÓN: Dibujar circunferencia para 1 axis, con pi/2 (ángulo recto) que representa una rotacion en pi

%PIZARRÓN: Dibujar circunferencia para 2 axis, con pi/2 en x que representa una rotación en pi, e ídem para y

% Bueno, cuando podemos rotar en 3 ejes distintos necesitamos una esfera de 4 dimensiones y no la vamos a poder dibujar, así que analicemos el caso en el que solo nos interesa rotar un punto de $\R^3$ pero alrededor del eje x o el eje y (o los nombres que quieran)

% (En la esfera de 4 dimensiones la idea es la misma pero no lo podríamos hacer de manera gráfica)

% Si los ángulos no se doblaran, tendríamos que rotar \pi grados alrededor de un eje daría lo mismo que \pi grados alrededor del otro! Pero esto no es cierto

%TODO Es cierto que un cuaternión que está en la esfera pero rotado en x representa una rotación alrededor del eje x?

\end{frame}

\iffalse
\begin{frame}{Los cuaterniones se pueden llevar a matrices}

Dado un cuaternión $q=q_0 + q_1 \textbf{i} + q_2 \textbf{j} + q_3 \textbf{k}$, la rotación que representa se puede dar con: 

\[
A = \begin{bmatrix}
    (q_0^2+q_1^2-q_2^2-q_3^2)  &  2(q_1 q_2 - q_0 q_3) & 2(q_1 q_3 + q_0 q_2)      \\
    2(q_1 q_2 + q_0 q_3) & (q_0^2 - q_1^2 + q_2^2 - q_3^2) & 2(q_2 q_3 - q_0 q_1) \\
    2(q_1 q_3 - q_0 q_2) & 2(q_2 q_3 + q_0 q_1) & (q_0^2 - q_1^2 - q_2^2 + q_3^2)
\end{bmatrix}
\]

%TODO cambiar a a,b,c,d?


$$v \mapsto A\cdot v$$
%\includegraphics[scale=0.5]{quaternion_matrix.png}
\end{frame}

\fi

\iffalse
\begin{frame}{Frame Title}
    Hablar de exponenciación de raeles y complejos
    Con QuaternionReport1 puedo definir el Slerp como una exponenciación.
    
    Visualizing Quaternions pagina 102
    
    Usamos el producto para mantener módulo $1$.
\end{frame}



\begin{frame}{SLERP (interpolación a velocidad constante)}

\end{frame}

\begin{frame}{SLERP (interpolación a velocidad constante)}

%TODO q y -q representan la misma rotación
\Huge{TODO decir que el slerp puede ir por el camino largo o corto} %http://caig.cs.nctu.edu.tw/course/CA/Lecture/slerp.pdf
%Justamente la diferencia entre q y -q se da en el SLERP :D


¿Qué significa esto? Creo que es elegir el segundo cuaternión con el signo para que el producto de positivo, ver link. Eso hace que el camino sea el más corto.
% long way y short way
% https://en.wikipedia.org/wiki/Slerp#Quaternion_Slerp
%Ah! es que la rotación en el plano que te queda te la vuelta corta o la vuelta larga.

%Los cuaterniones son puntos de $R^4$ de módulo 1: están en una especie de esfera de 4 dimensiones que no nos podemos imaginar.

¿Cómo hacemos si queremos rotar un punto de $v$ a $qvq\m$ pero de manera 'continua', a velocidad constante? O, más difícil, si tenemos un vector $v$ ya rotado, o sea, tenemos $p v p\m$ y lo queremos rotar a $q v q\m$? Esto generaliza el caso anterior tomando $p = 1$ \pause

¡Movemos \textit{los cuaterniones}!

% Tienen que re tener en la cabeza que tienen módulo 1

% Creo que es así: lo que uno hace es mover el cuaternión por la esfera. % Esfera de cuatro dimensiones

%Imaginémonos por un segundo que no nos queremos mover por una esfera sino por 'numeritos' (ojo que no son puntos en esta simplificación). Allá lejos y hace tiempo, en la secundaria, aprendimos a hacer esto. Es una función lineal que pasa por dos puntos.




\end{frame}

\fi


\begin{frame}{SLERP: interpolación a velocidad constante}

\includegraphics[scale=0.66]{Slerp2.jpg} \bigskip

¿Cómo hacemos si queremos rotar un punto $v$ a otro punto llamado $w$ a velocidad constante? \pause

Necesariamente $w$ debe ser $v$ rotado, por un quaternión $q$. O sea, $\hat{w}=q\hat{v}q\m$. \pause

\end{frame}

\begin{frame}{SLERP: interpolación a velocidad constante (cont)}

Resolvamos un problema más 'genérico' (es más fácil así):

Tenemos $p\hat{v}p\m$ y lo queremos mover a $q\hat{v}q\m$.

(Si tomamos $p=1$ recuperamos el problema original) \pause

¿Cómo hacemos? \pause

¡La idea va a ser olvidarnos del vector $v$ y mover los cuaterniones!

\end{frame}

\begin{frame}{Primero con numeritos}

Queremos mover $p$ a $q$. Finjamos primero que no son cuaterniones, sino numeritos reales. \pause


%PIZARRÓN: función lineal con eje x igual al tiempo y eje y igual 

% Es un sistema de ecuaciones lineales, con dos ecuaciones y dos incógnitas, sarasa

$$f(t) = (1-t) \cdot p + t\cdot q$$

$f'(t) = - p + q$: se mueve a velocidad constante. \pause \bigskip


Nota: la misma formulita en 3D anda bien.

% En 0 da p, en 1 da q
%PIZARRÓN: la fórmula

\end{frame}

\begin{frame}{Ahora en la circunferencia}

% PIZARRÓN dibujarla

\begin{figure}  		
  	\centering
	\includegraphics[scale=0.5]{slerpR2_p.png}
\end{figure}

Quiero una función $f(t)$ tal que $f(0) = p, f(1) = q$, y además $f'$ constante (para tener velocidad constante)

$f(t) = (1-t) \cdot p + t\cdot q$ anda en el sentido de que también se mueve a velocidad constante.

Pero...no se queda en la circunferencia.

%TODO dibujo


\end{frame}


\begin{frame}{Ahora en la circunferencia (II)} %TODO titulo
$$f(t) = (1-t)\cdot p + t\cdot q$$

\begin{figure}  		
  	\centering
	\includegraphics[scale=0.5]{slerpR2_2_p.png}
\end{figure} \pause

¿Cómo hacemos para que vaya por la circunferencia? \pause

\end{frame}

\begin{frame}{Ahora en la circunferencia(III)}

\begin{itemize}
	\item Los puntos de la circunferencia son los que tienen módulo $1$
	\item Si $(x,y) \in \R^2$, entonces $\frac{(x,y)}{||(x,y)||}$ tiene módulo 1, y está en la circunferencia.
\end{itemize}

$$f(t) = \frac{(1-t) \cdot p + t\cdot q}{||(1-t)\cdot p+ t\cdot q||}$$

\begin{figure}
  	\centering
	\includegraphics[scale=0.44]{slerpR2_3_p.png}
\end{figure}

Pero...no va a velocidad constante. Hay que arreglar un poco las cosas.

\end{frame}

\begin{frame}{¿Y con cuaterniones?}

%TODO agregar negrita para q_2 y q_2


$$(1-t) \cdot p + t \cdot q$$

Una vez arreglada la función para que la velocidad quede igual a $1$, sirve para puntos en el plano, y también cuaterniones:

$$Slerp(p,q,t) = \frac{sen(\textbf{(1-t)}\theta)}{sen(\theta)} p + \frac{sen(\textbf{t}\theta)}{sen(\theta)} q$$ \pause

¡Fíjense la simetría que tiene!

%PIZARRÓN: mancha fea SOBRE módulo de mancha fea.

($\theta$ se obtiene a partir de $p$ y $q$, es 'el ángulo' entre ellos)

%TODO SLERP es y para que se usa

\end{frame}


\begin{frame}{Volvamos a interpolación de numeritos}

$$f(t) = (1-t) \cdot p + t \cdot q$$

Es lo mismo que:

$$f(t) = p-t\cdot p + t\cdot q = p + (-p+q) \cdot t$$ \pause

Esa idea se puede trasladar a cuaterniones, miremos cómo:

$$f(t)= p (p^*q)^t$$ \pause

donde $\textit{elevar a la $t$}$ es algo parecido a la exponenciación que todos conocemos de los números reales. \pause

Y como al multiplicar dos cuaterniones se multiplican los módulos, todo anda bien.

\end{frame}

\begin{frame}{Resumencito III}

\begin{itemize}
    \item Los cuaterniones son un upgrade de los complejos.
    \item Tienen más letras y eso les hace perder la conmutatividad.
    \item Necesitamos agregar dos letras más porque las necesitamos para 'describir' el $\textbf{eje}$ de rotación y el \textbf{ángulo}
    \item Rotar con cuaterniones es como complejos pero un poco más feo, multiplicando a izquierda y a derecha por $q$ y haciendo un par más de cosas
    \item Como multiplicamos dos veces por $q$, si queremos rotar en un ángulo $\theta$, al cuaternión lo inventamos con $\theta/2$.
    \item Así como a los complejos se los puede describir con la longitud y el ángulo, a los cuaterniones se los puede describir con 'la normal' (o sea, 'el mango'), y el ángulo. Y de esa manera la rotación queda bien a la vista.
    \item Para rotar puntos a velocidad consante, rotamos los cuaterniones correspondientes a velocidad constante.
    
\end{itemize}
    
\end{frame}

\begin{frame}{Fin}

% TODO imagen graciosa de rotaciones
% Girando, girando hacia la libertad
\includegraphics[scale=0.4]{chiste2.jpg} \\
\Huge{¿Preguntas?}
\end{frame}



\iffalse


\begin{frame}{Cuaterniones como rotaciones}
¿Por qué necesitamos 3 letritas y no basta con 2? ¿Hago la cuenta?

Unir esto con "dimensiones": necesito $n$ grados de libertad porque sarasa.

OJO que tienen que ser unitarios los cuaterniones, creo.

\end{frame}

\begin{frame}{Regla de la mano derecha}
\end{frame}

\begin{frame}
	Cuaterniones: 'la suma de un escalar y un vector'
	El producto de cuaterniones se define con la doble distributiva y las reglas de la slide que vimos hace un rato.
	¿Cómo actúa un cuaternión sobre un vector de $\R^3$? Un vector de $\R^3$ se puede ver simplemente como un cuaternión tal que su parte real es $0$. Me parece que vamos a terminar multiplicándolos y listo.

	
\end{frame}

\begin{frame}{Resumencito}

\end{frame}

\begin{frame}

\end{frame}

\begin{frame}{¿Preguntas?}

\end{frame}




%---------------------------- "Borrador de charla" ----------------------------%

\begin{frame}{Resumen}
	La idea de la charla es hablar de cuaterniones. Los cuaterniones son una especie de números complejos en $\R^4$ (con $i,j,k$ en vez de sólo $i$), así que va a convenir hablar primero de números complejos. Además, los números complejos representan rotaciones en $\R^2$, así que va a convenir empezar por ahí.


\end{frame}


\begin{frame}{Rotaciones en $\mathbb{R}^2$}
 Números complejos al ataque.
 
 Regla: los módulos se multiplican y los ángulos se suman.
 
 NO hablar de forma trigonométrica (¿se llamaba así o me estoy confundiendo?) salvo que me sobre tiempo.
 
 Estaría bueno hacer algunas cuentitas, para entender eso de los ángulos, en realidad.
 
 Observar que para rotaciones vamos a necesitar módulo $1$.
\end{frame}

\begin{frame}
	¿Qué es un número complejo? ¿Cómo se representan en el plano complejo? ¿Qué es su ángulo? ¿Y su módulo?

	Cuando dos números complejos se multiplican, se suman los ángulos y se multiplican los módulos. Ejemplo: $(3+4i) \cdot (1+i) = 3-4 + (4+3)i = -1 +7i$; los módulos eran $5,\sqrt{2}$ y $\sqrt{50}$. Los ángulos salen de un dibujito, fue. Puedo poner $(1+i)$ rotado, pegado, al lado de $3+4i$ para que se note.
	
	Fíjense que con esta operación el $0$ sigue el elemento absorbente, el $1$ sigue siendo el neutro, todo bien. (Decirlo en criollo). Y si multiplicamos dos números reales es lo mismo de siempre, etc.	
	
	
	Las rotaciones con centro en el origen (se dice así) entonces, pueden ser dadas por números complejos de módulo $1$. ¿Cómo conocemos el ángulo de un número complejo? Con trigonometría. Senos y cosenos. Y obtenemos $z=a+bi = |z| (cos(\theta) + i sen(\theta)$.

Y fíjense qué lindo, la inversa de una rotación es simplemente usar $2\pi-\theta$, y la composición de dos rotaciones es multiplicar los números complejos correspondientes, etc.
\end{frame}

\begin{frame}
	%TODO resaltar con distintos colores
	
	\includegraphics[scale=1]{R2.png}
\end{frame}

\begin{frame}
	No sé si hacer un ejemplo de a partir de dos vectores conseguir la rotación correspondiente. Creo que sí.
\end{frame}

\begin{frame}
	Preguntas para pensar:
	
	\begin{itemize}
		\item ¿Qué rotación me lleva el $1$ al $i$? Pensar qué ángulo necesito
	\end{itemize}
\end{frame}


\begin{frame}{Matrices}
	Ahora bien, todo esto tiene que ver con matrices.
	
	Recordar producto de matrices, y escribir matrices en latex :-)

Si $z=a+bi$, funciona tratar a los complejos como si fueran matrices. Así:

$$a, -b$$
$$b, a$$

Pero haciéndolo con ángulos y determinante $1$, queda:

$$cos(\theta), -sen(\theta)$$
$$sen(\theta), cos(\theta)$$

donde la matriz es re especial: $A^2 = Id, det(A)=1$.



\end{frame}


\begin{frame}{¿Qué es una rotación (en $\R^3$)?}
	\includegraphics[scale=0.6]{rotation.png}
	
	Una rotación está compuesta por un eje $\ell$ (que define un plano ortogonal) y un ángulo $\theta$.
	
	Para cada plano ortogonal, lo que tenemos es una rotación en $\R^2$! Lo que no estoy seguro es de dónde sale el ángulo de rotación. Será el del cuaternión? Será bien visible?
\end{frame}

\begin{frame}{Ángulos de Euler}
	Los ángulos de Euler no sé lo que son exactamente, pero se usan rotaciones a través de los 3 ejes (y eso basta para representar cualquier rotación)	
	%TODO ver cómo es el teorema!
	
	Lo malo es que no es única la descomposición. Hagamos algún ejemplo.
	%TODO hacer ejemplo de dos maneras distintas.
	
	¡Además importa el orden! En ese sentido las rotaciones en $R^3$ no son conmutativas (mientras que en $R^2$ sí). Llevar cubo rubik.
\end{frame}

\begin{frame}{Ángulos de Euler (cont)}
	Los ángulos de Euler se pueden representar con matrices. ¿Recuerdan cómo operar con matrices? Hagamos algún ejemplo.
	
	Entonces, una rotación usando ángulos de Euler se pueden hacer multiplicando con 3 matrices distintas. No sé si recuerdan pero el producto de matrices en general no es conmutativo. En el apéndice $B$ de QuaternionReport está la rotación en cada eje, y se pueden usar de ejemplo.
\end{frame}

\begin{frame}{Cuaterniones}
	Los cuaterniones fueron desarrollados por Hamilton en un intento de generalizar $\C$ para que sea aplicable a $\R^3$. El tipo pensaba que bastar una coordenada imaginaria más, llamémosla $j$. Long story short, eso no se puede. Pero (literalmente) un día estaba caminando y se le ocurrió probar con 4 dimensiones y se imaginó las reglas que tenía que usar y las escribió en un puente y sarasa.
\end{frame}

\begin{frame}
	Lo que escribió en el puente fue esto:
	
	\includegraphics[scale=0.5]{hamilton.png}
	%$$i^2 = j^2 = k^2 = ijk = −1$$
	
	Y yo les podría decir que los cuaterniones son super parecidos a los complejos pero con dos letritas más y reglas adicionales....y lo son, pero hay que meterse de a poco en como \textit{operar} con ellos.
	
	La formulita escribirla en un papel.
\end{frame}


\begin{frame}{No sé si contar esto: por qué no anda con 3 dimensiones}

La idea es que un cuaternión represente una rotación. Y uno se puede preguntar por qué, dado que dos variables (números complejos) sirven para representar rotaciones en $\R^2$, no se usan 3 variables para rotaciones en $\R^3$.


%TODO arreglar las comillas
Pero supongamos que anda esto de usar 3 variables  Las podríamos representar con un vector $(x,y,z)$ o con "números" de la forma $a+b\ii+c\jj$. \\

Hamilton quería que $i^2=j^2=-1$ para generalizar a los números complejos.


¿Qué pasa cuando multiplicamos $\ii$ por $\jj$? Da...algo. Digamos que deben existir $a,b,c\in\R$ tal que $\ii\jj=a+b\ii+c\jj$. Pero multiplicando a cada lado por $\ii$ obtenemos $-j = a\ii-b+c\ii\jj$. Pero...ya sabemos cuánto vale $\ii\jj$. Reemplazando, obtenemos lo siguiente:

$-\jj=a\ii-b+c(a+b\ii+c\jj)$, o sea, $0=(ac-b)+\ii(a+bc)+\jj(c^2+1)$. Y esto implica que $c^2+1 = 0$, lo cuál no puede ser.


\end{frame}

\begin{frame}
	En algún momento quiero mencionar que vamos a trabajar con unit-length quaternions, o sea, en una esfera en $\R^4$, y que por lo tanto tienen solo 3 grados de libertad.
	
	¿Es verdad que necesito 4 numeritos para dar el eje y el ángulo? Para mí sí, total el eje lo puedo dividir por cualquier número para normalizar el cuaternión...igual no estoy tan seguro. Pero si esto es verdad tengo que decidir entre si contarlo, o lo de los 3 grados de libertad, porque ambas cosas confunden.
\end{frame}

\begin{frame}{Cuaterniones (cont)}
	Cuando hable de ellos tratarlos como ángulo + vector de $\R^3$. Y notar que eso se corresponde bien con una rotación!
\end{frame}


\begin{frame}{Gimbal lock}
Mencionar el incidente de Apollo, página 23 de Visualizing Quaternions.djvu. Incluir imágenes


\end{frame}

\begin{frame}
Podría hablar acá de diferencias entre rotaciones, ángulos de Euler y cuaterniones.

Ver página 30 del quaternion report. Igual no me cabe mucho la idea.

Idea: hacer una tabla, y poner que la gran desventaja de los cuaterniones es que son "complicados", y actualizar la slide marcando en rojo y diciendo "bueno, tratemos de eliminar esto"
\end{frame}

\fi

\end{document}
